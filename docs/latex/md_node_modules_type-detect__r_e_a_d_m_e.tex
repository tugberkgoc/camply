\section*{\href{http://chaijs.com}{\tt type-\/detect } }

~\newline
 

Improved typeof detection for \href{http://nodejs.org}{\tt node} and the browser. 

\href{./LICENSE}{\tt } \href{https://github.com/chaijs/type-detect/releases}{\tt } \href{https://travis-ci.org/chaijs/type-detect}{\tt } \href{https://coveralls.io/r/chaijs/type-detect}{\tt } \href{https://www.npmjs.com/packages/type-detect}{\tt } \href{https://www.npmjs.com/packages/type-detect}{\tt } \href{}{\tt } ~\newline
 \tabulinesep=1mm
\begin{longtabu} spread 0pt [c]{*{6}{|X[-1]}|}
\hline
\rowcolor{\tableheadbgcolor}\multicolumn{6}{|p{(\linewidth-\tabcolsep*6-\arrayrulewidth*1)*6/6}|}{\cellcolor{\tableheadbgcolor}\textbf{ Supported Browsers }}\\\cline{1-6}
\endfirsthead
\hline
\endfoot
\hline
\rowcolor{\tableheadbgcolor}\multicolumn{6}{|p{(\linewidth-\tabcolsep*6-\arrayrulewidth*1)*6/6}|}{\cellcolor{\tableheadbgcolor}\textbf{ Supported Browsers }}\\\cline{1-6}
\endhead
\rowcolor{\tableheadbgcolor}\PBS\centering \textbf{  Chrome }&\PBS\centering \textbf{  Edge }&\PBS\centering \textbf{  Firefox }&\PBS\centering \textbf{  Safari }&\PBS\centering \textbf{  IE }\\\cline{1-6}
\PBS\centering ✅ &\PBS\centering ✅ &\PBS\centering ✅ &\PBS\centering ✅ &\PBS\centering 9, 10, 11  \\\cline{1-6}
\end{longtabu}
~\newline
 \href{https://chai-slack.herokuapp.com/}{\tt } \href{https://gitter.im/chaijs/chai}{\tt } 

\subsection*{What is Type-\/\+Detect?}

Type Detect is a module which you can use to detect the type of a given object. It returns a string representation of the object\textquotesingle{}s type, either using \href{http://www.ecma-international.org/ecma-262/6.0/index.html#sec-typeof-operator}{\tt {\ttfamily typeof}} or \href{http://www.ecma-international.org/ecma-262/6.0/index.html#sec-symbol.tostringtag}{\tt {\ttfamily @@to\+String\+Tag}}. It also normalizes some object names for consistency among browsers.

\subsection*{Why?}

The {\ttfamily typeof} operator will only specify primitive values; everything else is {\ttfamily \char`\"{}object\char`\"{}} (including {\ttfamily null}, arrays, regexps, etc). Many developers use {\ttfamily Object.\+prototype.\+to\+String()} -\/ which is a fine alternative and returns many more types (null returns {\ttfamily \mbox{[}object Null\mbox{]}}, Arrays as {\ttfamily \mbox{[}object Array\mbox{]}}, regexps as {\ttfamily \mbox{[}object Reg\+Exp\mbox{]}} etc).

Sadly, {\ttfamily Object.\+prototype.\+to\+String} is slow, and buggy. By slow -\/ we mean it is slower than {\ttfamily typeof}. By buggy -\/ we mean that some values (like Promises, the global object, iterators, dataviews, a bunch of H\+T\+ML elements) all report different things in different browsers.

{\ttfamily type-\/detect} fixes all of the shortcomings with {\ttfamily Object.\+prototype.\+to\+String}. We have extra code to speed up checks of JS and D\+OM objects, as much as 20-\/30x faster for some values. {\ttfamily type-\/detect} also fixes any consistencies with these objects.

\subsection*{Installation}

\subsubsection*{Node.\+js}

{\ttfamily type-\/detect} is available on \href{http://npmjs.org}{\tt npm}. To install it, type\+: \begin{DoxyVerb}$ npm install type-detect
\end{DoxyVerb}


\subsubsection*{Browsers}

You can also use it within the browser; install via npm and use the {\ttfamily type-\/detect.\+js} file found within the download. For example\+:


\begin{DoxyCode}
<script src="./node\_modules/type-detect/type-detect.js"></script>
\end{DoxyCode}


\subsection*{Usage}

The primary export of {\ttfamily type-\/detect} is function that can serve as a replacement for {\ttfamily typeof}. The results of this function will be more specific than that of native {\ttfamily typeof}.


\begin{DoxyCode}
var type = require('type-detect');
\end{DoxyCode}


\paragraph*{array}


\begin{DoxyCode}
assert(type([]) === 'Array');
assert(type(new Array()) === 'Array');
\end{DoxyCode}


\paragraph*{regexp}


\begin{DoxyCode}
assert(type(/a-z/gi) === 'RegExp');
assert(type(new RegExp('a-z')) === 'RegExp');
\end{DoxyCode}


\paragraph*{function}


\begin{DoxyCode}
assert(type(function () \{\}) === 'function');
\end{DoxyCode}


\paragraph*{arguments}


\begin{DoxyCode}
(function () \{
  assert(type(arguments) === 'Arguments');
\})();
\end{DoxyCode}


\paragraph*{date}


\begin{DoxyCode}
assert(type(new Date) === 'Date');
\end{DoxyCode}


\paragraph*{number}


\begin{DoxyCode}
assert(type(1) === 'number');
assert(type(1.234) === 'number');
assert(type(-1) === 'number');
assert(type(-1.234) === 'number');
assert(type(Infinity) === 'number');
assert(type(NaN) === 'number');
assert(type(new Number(1)) === 'Number'); // note - the object version has a capital N
\end{DoxyCode}


\paragraph*{string}


\begin{DoxyCode}
assert(type('hello world') === 'string');
assert(type(new String('hello')) === 'String'); // note - the object version has a capital S
\end{DoxyCode}


\paragraph*{null}


\begin{DoxyCode}
assert(type(null) === 'null');
assert(type(undefined) !== 'null');
\end{DoxyCode}


\paragraph*{undefined}


\begin{DoxyCode}
assert(type(undefined) === 'undefined');
assert(type(null) !== 'undefined');
\end{DoxyCode}


\paragraph*{object}


\begin{DoxyCode}
var Noop = function () \{\};
assert(type(\{\}) === 'Object');
assert(type(Noop) !== 'Object');
assert(type(new Noop) === 'Object');
assert(type(new Object) === 'Object');
\end{DoxyCode}


\paragraph*{E\+C\+M\+A6 Types}

All new E\+C\+M\+A\+Script 2015 objects are also supported, such as Promises and Symbols\+:


\begin{DoxyCode}
assert(type(new Map() === 'Map');
assert(type(new WeakMap()) === 'WeakMap');
assert(type(new Set()) === 'Set');
assert(type(new WeakSet()) === 'WeakSet');
assert(type(Symbol()) === 'symbol');
assert(type(new Promise(callback) === 'Promise');
assert(type(new Int8Array()) === 'Int8Array');
assert(type(new Uint8Array()) === 'Uint8Array');
assert(type(new UInt8ClampedArray()) === 'Uint8ClampedArray');
assert(type(new Int16Array()) === 'Int16Array');
assert(type(new Uint16Array()) === 'Uint16Array');
assert(type(new Int32Array()) === 'Int32Array');
assert(type(new UInt32Array()) === 'Uint32Array');
assert(type(new Float32Array()) === 'Float32Array');
assert(type(new Float64Array()) === 'Float64Array');
assert(type(new ArrayBuffer()) === 'ArrayBuffer');
assert(type(new DataView(arrayBuffer)) === 'DataView');
\end{DoxyCode}


Also, if you use {\ttfamily Symbol.\+to\+String\+Tag} to change an Objects return value of the {\ttfamily to\+String()} Method, {\ttfamily type()} will return this value, e.\+g\+:


\begin{DoxyCode}
var myObject = \{\};
myObject[Symbol.toStringTag] = 'myCustomType';
assert(type(myObject) === 'myCustomType');
\end{DoxyCode}
 