\begin{quote}
Create error class \end{quote}


\subsection*{Install}


\begin{DoxyCode}
$ npm install --save create-error-class
\end{DoxyCode}


\subsection*{Usage}


\begin{DoxyCode}
var createErrorClass = require('create-error-class');

var HTTPError = createErrorClass('HTTPError', function (props) \{
  this.message = 'Status code is ' + props.statusCode;
\});

throw new HTTPError(\{statusCode: 404\});
\end{DoxyCode}


\subsection*{A\+PI}

\subsubsection*{create\+Error\+Class(class\+Name, \mbox{[}setup\mbox{]})}

Return constructor of Errors with {\ttfamily class\+Name}.

\paragraph*{class\+Name}

{\itshape Required} ~\newline
Type\+: {\ttfamily string}

Class name of Error Object. Should contain characters from {\ttfamily \mbox{[}0-\/9a-\/z\+A-\/\+Z\+\_\+\$\mbox{]}} range.

\paragraph*{setup}

Type\+: {\ttfamily function}

Setup function, that will be called after each Error object is created from constructor with context of Error object.

By default {\ttfamily setup} function sets {\ttfamily this.\+message} as first argument\+:


\begin{DoxyCode}
var MyError = createErrorClass('MyError');

new MyError('Something gone wrong!').message; // => 'Something gone wrong!'
\end{DoxyCode}


\subsection*{License}

M\+IT © \href{http://github.com/floatdrop}{\tt Vsevolod Strukchinsky} 