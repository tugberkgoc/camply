\href{https://npmjs.org/package/destroy}{\tt } \href{https://travis-ci.org/stream-utils/destroy}{\tt } \href{https://coveralls.io/r/stream-utils/destroy?branch=master}{\tt } !\mbox{[}License\mbox{]}\mbox{[}license-\/image\mbox{]} \href{https://npmjs.org/package/destroy}{\tt } \href{https://www.gittip.com/jonathanong/}{\tt }

Destroy a stream.

This module is meant to ensure a stream gets destroyed, handling different A\+P\+Is and Node.\+js bugs.

\subsection*{A\+PI}


\begin{DoxyCode}
var destroy = require('destroy')
\end{DoxyCode}


\subsubsection*{destroy(stream)}

Destroy the given stream. In most cases, this is identical to a simple {\ttfamily stream.\+destroy()} call. The rules are as follows for a given stream\+:


\begin{DoxyEnumerate}
\item If the {\ttfamily stream} is an instance of {\ttfamily Read\+Stream}, then call {\ttfamily stream.\+destroy()} and add a listener to the {\ttfamily open} event to call {\ttfamily stream.\+close()} if it is fired. This is for a Node.\+js bug that will leak a file descriptor if {\ttfamily .destroy()} is called before {\ttfamily open}.
\item If the {\ttfamily stream} is not an instance of {\ttfamily Stream}, then nothing happens.
\item If the {\ttfamily stream} has a {\ttfamily .destroy()} method, then call it.
\end{DoxyEnumerate}

The function returns the {\ttfamily stream} passed in as the argument.

\subsection*{Example}


\begin{DoxyCode}
var destroy = require('destroy')

var fs = require('fs')
var stream = fs.createReadStream('package.json')

// ... and later
destroy(stream)
\end{DoxyCode}
 