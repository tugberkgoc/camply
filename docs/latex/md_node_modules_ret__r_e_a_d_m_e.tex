Tokenizes strings that represent a regular expressions.

\href{http://travis-ci.org/fent/ret.js}{\tt } \href{https://david-dm.org/fent/ret.js}{\tt } \href{https://codecov.io/gh/fent/ret.js}{\tt }

\section*{Usage}


\begin{DoxyCode}
var ret = require('ret');

var tokens = ret(/foo|bar/.source);
\end{DoxyCode}


{\ttfamily tokens} will contain the following object


\begin{DoxyCode}
\{
  "type": ret.types.ROOT
  "options": [
    [ \{ "type": ret.types.CHAR, "value", 102 \},
      \{ "type": ret.types.CHAR, "value", 111 \},
      \{ "type": ret.types.CHAR, "value", 111 \} ],
    [ \{ "type": ret.types.CHAR, "value",  98 \},
      \{ "type": ret.types.CHAR, "value",  97 \},
      \{ "type": ret.types.CHAR, "value", 114 \} ]
  ]
\}
\end{DoxyCode}


\section*{Token Types}

{\ttfamily ret.\+types} is a collection of the various token types exported by ret.

\subsubsection*{R\+O\+OT}

Only used in the root of the regexp. This is needed due to the posibility of the root containing a pipe {\ttfamily $\vert$} character. In that case, the token will have an {\ttfamily options} key that will be an array of arrays of tokens. If not, it will contain a {\ttfamily stack} key that is an array of tokens.


\begin{DoxyCode}
\{
  "type": ret.types.ROOT,
  "stack": [token1, token2...],
\}
\end{DoxyCode}



\begin{DoxyCode}
\{
  "type": ret.types.ROOT,
  "options" [
    [token1, token2...],
    [othertoken1, othertoken2...]
    ...
  ],
\}
\end{DoxyCode}


\subsubsection*{G\+R\+O\+UP}

Groups contain tokens that are inside of a parenthesis. If the group begins with {\ttfamily ?} followed by another character, it\textquotesingle{}s a special type of group. A \textquotesingle{}\+:\textquotesingle{} tells the group not to be remembered when {\ttfamily exec} is used. \textquotesingle{}=\textquotesingle{} means the previous token matches only if followed by this group, and \textquotesingle{}!\textquotesingle{} means the previous token matches only if N\+OT followed.

Like root, it can contain an {\ttfamily options} key instead of {\ttfamily stack} if there is a pipe.


\begin{DoxyCode}
\{
  "type": ret.types.GROUP,
  "remember" true,
  "followedBy": false,
  "notFollowedBy": false,
  "stack": [token1, token2...],
\}
\end{DoxyCode}



\begin{DoxyCode}
\{
  "type": ret.types.GROUP,
  "remember" true,
  "followedBy": false,
  "notFollowedBy": false,
  "options" [
    [token1, token2...],
    [othertoken1, othertoken2...]
    ...
  ],
\}
\end{DoxyCode}


\subsubsection*{P\+O\+S\+I\+T\+I\+ON}

{\ttfamily \textbackslash{}b}, {\ttfamily \textbackslash{}B}, {\ttfamily $^\wedge$}, and {\ttfamily \$} specify positions in the regexp.


\begin{DoxyCode}
\{
  "type": ret.types.POSITION,
  "value": "^",
\}
\end{DoxyCode}


\subsubsection*{S\+ET}

Contains a key {\ttfamily set} specifying what tokens are allowed and a key {\ttfamily not} specifying if the set should be negated. A set can contain other sets, ranges, and characters.


\begin{DoxyCode}
\{
  "type": ret.types.SET,
  "set": [token1, token2...],
  "not": false,
\}
\end{DoxyCode}


\subsubsection*{R\+A\+N\+GE}

Used in set tokens to specify a character range. {\ttfamily from} and {\ttfamily to} are character codes.


\begin{DoxyCode}
\{
  "type": ret.types.RANGE,
  "from": 97,
  "to": 122,
\}
\end{DoxyCode}


\subsubsection*{R\+E\+P\+E\+T\+I\+T\+I\+ON}


\begin{DoxyCode}
\{
  "type": ret.types.REPETITION,
  "min": 0,
  "max": Infinity,
  "value": token,
\}
\end{DoxyCode}


\subsubsection*{R\+E\+F\+E\+R\+E\+N\+CE}

References a group token. {\ttfamily value} is 1-\/9.


\begin{DoxyCode}
\{
  "type": ret.types.REFERENCE,
  "value": 1,
\}
\end{DoxyCode}


\subsubsection*{C\+H\+AR}

Represents a single character token. {\ttfamily value} is the character code. This might seem a bit cluttering instead of concatenating characters together. But since repetition tokens only repeat the last token and not the last clause like the pipe, it\textquotesingle{}s simpler to do it this way.


\begin{DoxyCode}
\{
  "type": ret.types.CHAR,
  "value": 123,
\}
\end{DoxyCode}


\subsection*{Errors}

ret.\+js will throw errors if given a string with an invalid regular expression. All possible errors are


\begin{DoxyItemize}
\item Invalid group. When a group with an immediate {\ttfamily ?} character is followed by an invalid character. It can only be followed by {\ttfamily !}, {\ttfamily =}, or {\ttfamily \+:}. Example\+: {\ttfamily /(?\+\_\+abc)/}
\item Nothing to repeat. Thrown when a repetitional token is used as the first token in the current clause, as in right in the beginning of the regexp or group, or right after a pipe. Example\+: {\ttfamily /foo$\vert$?bar/}, {\ttfamily /\{1,3\}foo$\vert$bar/}, {\ttfamily /foo(+bar)/}
\item Unmatched ). A group was not opened, but was closed. Example\+: {\ttfamily /hello)2u/}
\item Unterminated group. A group was not closed. Example\+: {\ttfamily /(1(23)4/}
\item Unterminated character class. A custom character set was not closed. Example\+: {\ttfamily /\mbox{[}abc/}
\end{DoxyItemize}

\section*{Install}

\begin{DoxyVerb}npm install ret
\end{DoxyVerb}


\section*{Tests}

Tests are written with \href{http://vowsjs.org/}{\tt vows}


\begin{DoxyCode}
npm test
\end{DoxyCode}


\section*{License}

M\+IT 