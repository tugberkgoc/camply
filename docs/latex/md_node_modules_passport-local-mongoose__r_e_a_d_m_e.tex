Passport-\/\+Local Mongoose is a \href{http://mongoosejs.com/}{\tt Mongoose} \href{http://mongoosejs.com/docs/plugins.html}{\tt plugin} that simplifies building username and password login with \href{http://passportjs.org}{\tt Passport}.

\href{https://travis-ci.org/saintedlama/passport-local-mongoose}{\tt } \href{https://coveralls.io/r/saintedlama/passport-local-mongoose?branch=master}{\tt } \href{https://app.codellama.io/repositories/5a04399c1b4c363a0f9427f8}{\tt }

\subsection*{Tutorials}

Michael Herman gives a comprehensible walk through for setting up mongoose, passport, passport-\/local and passport-\/local-\/mongoose for user authentication in his blog post \href{http://mherman.org/blog/2013/11/11/user-authentication-with-passport-dot-js/}{\tt User Authentication With Passport.\+js}

\subsection*{Installation}


\begin{DoxyCode}
> npm install passport-local-mongoose
\end{DoxyCode}


Passport-\/\+Local Mongoose does not require {\ttfamily passport}, {\ttfamily passport-\/local} or {\ttfamily mongoose} dependencies directly but expects you to have these dependencies installed.

In case you need to install the whole set of dependencies


\begin{DoxyCode}
> npm install passport passport-local mongoose passport-local-mongoose --save
\end{DoxyCode}


\subsubsection*{Updating from 1.\+x to 2.\+x}

The default digest algorithm was changed due to security implications from {\bfseries sha1} to {\bfseries sha256}. If you decide to upgrade a production system from 1.\+x to 2.\+x your users {\bfseries will not be able to login} since the digest algorithm was changed! In these cases plan some migration strategy and/or use the {\bfseries sha1} option for the digest algorithm.

\subsection*{Usage}

\subsubsection*{Plugin Passport-\/\+Local Mongoose}

First you need to plugin Passport-\/\+Local Mongoose into your User schema


\begin{DoxyCode}
const mongoose = require('mongoose');
const Schema = mongoose.Schema;
const passportLocalMongoose = require('passport-local-mongoose');

const User = new Schema(\{\});

User.plugin(passportLocalMongoose);

module.exports = mongoose.model('User', User);
\end{DoxyCode}


You\textquotesingle{}re free to define your User how you like. Passport-\/\+Local Mongoose will add a username, hash and salt field to store the username, the hashed password and the salt value.

Additionally Passport-\/\+Local Mongoose adds some methods to your Schema. See the A\+PI Documentation section for more details.

\subsubsection*{Configure Passport/\+Passport-\/\+Local}

You should configure Passport/\+Passport-\/\+Local as described in \href{http://passportjs.org/guide/configure/}{\tt the Passport Guide}.

Passport-\/\+Local Mongoose supports this setup by implementing a {\ttfamily Local\+Strategy} and serialize\+User/deserialize\+User functions.

To setup Passport-\/\+Local Mongoose use this code


\begin{DoxyCode}
// requires the model with Passport-Local Mongoose plugged in
const User = require('./models/user');

// use static authenticate method of model in LocalStrategy
passport.use(new LocalStrategy(User.authenticate()));

// use static serialize and deserialize of model for passport session support
passport.serializeUser(User.serializeUser());
passport.deserializeUser(User.deserializeUser());
\end{DoxyCode}


Make sure that you have a mongoose connected to mongodb and you\textquotesingle{}re done.

\paragraph*{Simplified Passport/\+Passport-\/\+Local Configuration}

Starting with version 0.\+2.\+1 passport-\/local-\/mongoose adds a helper method {\ttfamily create\+Strategy} as static method to your schema. The {\ttfamily create\+Strategy} is responsible to setup passport-\/local {\ttfamily Local\+Strategy} with the correct options.


\begin{DoxyCode}
const User = require('./models/user');

// CHANGE: USE "createStrategy" INSTEAD OF "authenticate"
passport.use(User.createStrategy());

passport.serializeUser(User.serializeUser());
passport.deserializeUser(User.deserializeUser());
\end{DoxyCode}


The reason for this functionality is that when using the {\ttfamily username\+Field} option to specify an alternative username\+Field name, for example \char`\"{}email\char`\"{} passport-\/local would still expect your frontend login form to contain an input field with name \char`\"{}username\char`\"{} instead of email. This can be configured for passport-\/local but this is double the work. So we got this shortcut implemented.

\subsubsection*{Async/\+Await}

Starting with version {\ttfamily 5.\+0.\+0} passport-\/local-\/mongoose is async/await enabled by returning Promises for all instance and static methods except {\ttfamily serialize\+User} and {\ttfamily deserialize\+User}.


\begin{DoxyCode}
const user = new DefaultUser(\{username: 'user'\});
await user.setPassword('password');
await user.save();
const \{ user \} = await DefaultUser.authenticate()('user', 'password');
\end{DoxyCode}


\subsubsection*{Options}

When plugging in Passport-\/\+Local Mongoose plugin additional options can be provided to configure the hashing algorithm.


\begin{DoxyCode}
User.plugin(passportLocalMongoose, options);
\end{DoxyCode}


\paragraph*{Main Options}


\begin{DoxyItemize}
\item saltlen\+: specifies the salt length in bytes. Default\+: 32
\item iterations\+: specifies the number of iterations used in pbkdf2 hashing algorithm. Default\+: 25000
\item keylen\+: specifies the length in byte of the generated key. Default\+: 512
\item digest\+Algorithm\+: specifies the pbkdf2 digest algorithm. Default\+: sha256. (get a list of supported algorithms with crypto.\+get\+Hashes())
\item interval\+: specifies the interval in milliseconds between login attempts, which increases exponentially based on the number of failed attempts, up to max\+Interval. Default\+: 100
\item max\+Interval\+: specifies the maximum amount of time an account can be locked. Default 30000 (5 minutes)
\item username\+Field\+: specifies the field name that holds the username. Defaults to \textquotesingle{}username\textquotesingle{}. This option can be used if you want to use a different field to hold the username for example \char`\"{}email\char`\"{}.
\item username\+Unique \+: specifies if the username field should be enforced to be unique by a mongodb index or not. Defaults to true.
\item salt\+Field\+: specifies the field name that holds the salt value. Defaults to \textquotesingle{}salt\textquotesingle{}.
\item hash\+Field\+: specifies the field name that holds the password hash value. Defaults to \textquotesingle{}hash\textquotesingle{}.
\item attempts\+Field\+: specifies the field name that holds the number of login failures since the last successful login. Defaults to \textquotesingle{}attempts\textquotesingle{}.
\item last\+Login\+Field\+: specifies the field name that holds the timestamp of the last login attempt. Defaults to \textquotesingle{}last\textquotesingle{}.
\item select\+Fields\+: specifies the fields of the model to be selected from mongodb (and stored in the session). Defaults to \textquotesingle{}undefined\textquotesingle{} so that all fields of the model are selected.
\item username\+Lower\+Case\+: convert username field value to lower case when saving an querying. Defaults to \textquotesingle{}false\textquotesingle{}.
\item populate\+Fields\+: specifies fields to populate in find\+By\+Username function. Defaults to \textquotesingle{}undefined\textquotesingle{}.
\item encoding\+: specifies the encoding the generated salt and hash will be stored in. Defaults to \textquotesingle{}hex\textquotesingle{}.
\item limit\+Attempts\+: specifies whether login attempts should be limited and login failures should be penalized. Default\+: false.
\item max\+Attempts\+: specifies the maximum number of failed attempts allowed before preventing login. Default\+: Infinity.
\item password\+Validator\+: specifies your custom validation function for the password in the form \textquotesingle{}function(password,cb)\textquotesingle{}. Default\+: validates non-\/empty passwords.
\item username\+Query\+Fields\+: specifies alternative fields of the model for identifying a user (e.\+g. email).
\item find\+By\+Username\+: Specifies a query function that is executed with query parameters to restrict the query with extra query parameters. For example query only users with field \char`\"{}active\char`\"{} set to {\ttfamily true}. Default\+: {\ttfamily function(model, query\+Parameters) \{ return model.\+find\+One(query\+Parameters); \}}. See the examples section for a use case.
\end{DoxyItemize}

{\itshape Attention!} Changing any of the hashing options (saltlen, iterations or keylen) in a production environment will prevent that existing users to authenticate!

\paragraph*{Error Messages}

Override default error messages by setting options.\+error\+Messages.


\begin{DoxyItemize}
\item Missing\+Password\+Error \textquotesingle{}No password was given\textquotesingle{}
\item Attempt\+Too\+Soon\+Error \textquotesingle{}Account is currently locked. Try again later\textquotesingle{}
\item Too\+Many\+Attempts\+Error \textquotesingle{}Account locked due to too many failed login attempts\textquotesingle{}
\item No\+Salt\+Value\+Stored\+Error \textquotesingle{}Authentication not possible. No salt value stored\textquotesingle{}
\item Incorrect\+Password\+Error \textquotesingle{}Password or username are incorrect\textquotesingle{}
\item Incorrect\+Username\+Error \textquotesingle{}Password or username are incorrect\textquotesingle{}
\item Missing\+Username\+Error \textquotesingle{}No username was given\textquotesingle{}
\item User\+Exists\+Error \textquotesingle{}A user with the given username is already registered\textquotesingle{}
\end{DoxyItemize}

\subsubsection*{Hash Algorithm}

Passport-\/\+Local Mongoose use the pbkdf2 algorithm of the node crypto library. \href{http://en.wikipedia.org/wiki/PBKDF2}{\tt Pbkdf2} was chosen because platform independent (in contrary to bcrypt). For every user a generated salt value is saved to make rainbow table attacks even harder.

\subsubsection*{Examples}

For a complete example implementing a registration, login and logout see the \href{https://github.com/saintedlama/passport-local-mongoose/tree/master/examples/login}{\tt login example}.

\subsection*{A\+PI Documentation}

\subsubsection*{Instance methods}

\paragraph*{set\+Password(password, \mbox{[}cb\mbox{]})}

Sets a user password. Does not save the user object. If no callback {\ttfamily cb} is provided a {\ttfamily Promise} is returned.

\paragraph*{change\+Password(old\+Password, new\+Password, \mbox{[}cb\mbox{]})}

Changes a user\textquotesingle{}s password hash and salt and saves the user object. If no callback {\ttfamily cb} is provided a {\ttfamily Promise} is returned. If old\+Password does not match the user\textquotesingle{}s old password an {\ttfamily Incorrect\+Password\+Error} is passed to {\ttfamily cb} or the {\ttfamily Promise} is rejected.

\paragraph*{authenticate(password, \mbox{[}cb\mbox{]})}

Authenticate a user object. If no callback {\ttfamily cb} is provided a {\ttfamily Promise} is returned.

\paragraph*{reset\+Attempts(\mbox{[}cb\mbox{]})}

Reset a user\textquotesingle{}s number of failed password attempts (only defined if {\ttfamily options.\+limit\+Attempts} is true) If no callback {\ttfamily cb} is provided a {\ttfamily Promise} is returned.

\subsubsection*{Callback Arguments}


\begin{DoxyItemize}
\item err
\begin{DoxyItemize}
\item null unless the hasing algorithm throws an error
\end{DoxyItemize}
\item this\+Model
\begin{DoxyItemize}
\item the model getting authenticated {\itshape if} authentication was successful otherwise false
\end{DoxyItemize}
\item password\+Err
\begin{DoxyItemize}
\item an instance of {\ttfamily Authentication\+Error} describing the reason the password failed, else undefined.
\end{DoxyItemize}
\end{DoxyItemize}

Using {\ttfamily set\+Password()} will only update the document\textquotesingle{}s password fields, but will not save the document. To commit the changed document, remember to use Mongoose\textquotesingle{}s {\ttfamily document.\+save()} after using {\ttfamily set\+Password()}.

\subsubsection*{Error Handling}


\begin{DoxyItemize}
\item {\ttfamily Incorrect\+Password\+Error}\+: specifies the error message returned when the password is incorrect. Defaults to \textquotesingle{}Incorrect password\textquotesingle{}.
\item {\ttfamily Incorrect\+Username\+Error}\+: specifies the error message returned when the username is incorrect. Defaults to \textquotesingle{}Incorrect username\textquotesingle{}.
\item {\ttfamily Missing\+Username\+Error}\+: specifies the error message returned when the username has not been set during registration. Defaults to \textquotesingle{}Field s is not set\textquotesingle{}.
\item {\ttfamily Missing\+Password\+Error}\+: specifies the error message returned when the password has not been set during registration. Defaults to \textquotesingle{}Password argument not set!\textquotesingle{}.
\item {\ttfamily User\+Exists\+Error}\+: specifies the error message returned when the user already exists during registration. Defaults to \textquotesingle{}User already exists with name s\textquotesingle{}.
\item {\ttfamily No\+Salt\+Value\+Stored}\+: Occurs in case no salt value is stored in the Mongo\+DB collection.
\item {\ttfamily Attempt\+Too\+Soon\+Error}\+: Occurs if the option {\ttfamily limit\+Attempts} is set to true and a login attept occures while the user is still penalized.
\item {\ttfamily Too\+Many\+Attempts\+Error}\+: Returned when the user\textquotesingle{}s account is locked due to too many failed login attempts.
\end{DoxyItemize}

All those errors inherit from {\ttfamily Authentication\+Error}, if you need a more general error class for checking.

\subsubsection*{Static methods}

Static methods are exposed on the model constructor. For example to use create\+Strategy function use


\begin{DoxyCode}
const User = require('./models/user');
User.createStrategy();
\end{DoxyCode}



\begin{DoxyItemize}
\item authenticate() Generates a function that is used in Passport\textquotesingle{}s Local\+Strategy
\item serialize\+User() Generates a function that is used by Passport to serialize users into the session
\item deserialize\+User() Generates a function that is used by Passport to deserialize users into the session
\item register(user, password, cb) Convenience method to register a new user instance with a given password. Checks if username is unique. See \href{https://github.com/saintedlama/passport-local-mongoose/tree/master/examples/login}{\tt login example}.
\item find\+By\+Username() Convenience method to find a user instance by it\textquotesingle{}s unique username.
\item create\+Strategy() Creates a configured passport-\/local {\ttfamily Local\+Strategy} instance that can be used in passport.
\end{DoxyItemize}

\subsection*{Examples}

\subsubsection*{Allow only \char`\"{}active\char`\"{} users to authenticate}

First we define a schema with an additional field {\ttfamily active} of type Boolean.


\begin{DoxyCode}
var UserSchema = new Schema(\{
  active: Boolean
\});
\end{DoxyCode}


When plugging in Passport-\/\+Local Mongoose we set {\ttfamily username\+Unique} to avoid creating a unique mongodb index on field {\ttfamily username}. To avoid non active users to be queried by mongodb we can specify the option {\ttfamily find\+By\+Username} that allows us to restrict a query. In our case we want to restrict the query to only query users with field {\ttfamily active} set to {\ttfamily true}. The {\ttfamily find\+By\+Username} M\+U\+ST return a Mongoose query.


\begin{DoxyCode}
UserSchema.plugin(passportLocalMongoose, \{
  // Needed to set usernameUnique to true to avoid a mongodb index on the username column!
  usernameUnique: false,

  findByUsername: function(model, queryParameters) \{
    // Add additional query parameter - AND condition - active: true
    queryParameters.active = true;
    return model.findOne(queryParameters);
  \}
\});
\end{DoxyCode}


To test the implementation we can simply create (register) a user with field {\ttfamily active} set to {\ttfamily false} and try to authenticate this user in a second step\+:


\begin{DoxyCode}
var User = mongoose.model('Users', UserSchema);

User.register(\{username:'username', active: false\}, 'password', function(err, user) \{
  if (err) \{ ... \}

  var authenticate = User.authenticate();
  authenticate('username', 'password', function(err, result) \{
    if (err) \{ ... \}

    // Value 'result' is set to false. The user could not be authenticated since the user is not active
  \});
\});
\end{DoxyCode}


\subsection*{License}

Passport-\/\+Local Mongoose is licenses under the \href{http://opensource.org/licenses/MIT}{\tt M\+IT license}. 