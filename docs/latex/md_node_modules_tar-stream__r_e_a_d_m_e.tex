tar-\/stream is a streaming tar parser and generator and nothing else. It is streams2 and operates purely using streams which means you can easily extract/parse tarballs without ever hitting the file system.

Note that you still need to gunzip your data if you have a {\ttfamily .tar.\+gz}. We recommend using \href{https://github.com/mafintosh/gunzip-maybe}{\tt gunzip-\/maybe} in conjunction with this.


\begin{DoxyCode}
npm install tar-stream
\end{DoxyCode}


\href{http://travis-ci.org/mafintosh/tar-stream}{\tt } \href{http://opensource.org/licenses/MIT}{\tt }

\subsection*{Usage}

tar-\/stream exposes two streams, \href{https://github.com/mafintosh/tar-stream#packing}{\tt pack} which creates tarballs and \href{https://github.com/mafintosh/tar-stream#extracting}{\tt extract} which extracts tarballs. To \href{https://github.com/mafintosh/tar-stream#modifying-existing-tarballs}{\tt modify an existing tarball} use both.

It implementes U\+S\+T\+AR with additional support for pax extended headers. It should be compatible with all popular tar distributions out there (gnutar, bsdtar etc)

\subsection*{Related}

If you want to pack/unpack directories on the file system check out \href{https://github.com/mafintosh/tar-fs}{\tt tar-\/fs} which provides file system bindings to this module.

\subsection*{Packing}

To create a pack stream use {\ttfamily tar.\+pack()} and call {\ttfamily pack.\+entry(header, \mbox{[}callback\mbox{]})} to add tar entries.


\begin{DoxyCode}
var tar = require('tar-stream')
var pack = tar.pack() // pack is a streams2 stream

// add a file called my-test.txt with the content "Hello World!"
pack.entry(\{ name: 'my-test.txt' \}, 'Hello World!')

// add a file called my-stream-test.txt from a stream
var entry = pack.entry(\{ name: 'my-stream-test.txt', size: 11 \}, function(err) \{
  // the stream was added
  // no more entries
  pack.finalize()
\})

entry.write('hello')
entry.write(' ')
entry.write('world')
entry.end()

// pipe the pack stream somewhere
pack.pipe(process.stdout)
\end{DoxyCode}


\subsection*{Extracting}

To extract a stream use {\ttfamily tar.\+extract()} and listen for `extract.\+on(\textquotesingle{}entry', (header, stream, next) )\`{}


\begin{DoxyCode}
var extract = tar.extract()

extract.on('entry', function(header, stream, next) \{
  // header is the tar header
  // stream is the content body (might be an empty stream)
  // call next when you are done with this entry

  stream.on('end', function() \{
    next() // ready for next entry
  \})

  stream.resume() // just auto drain the stream
\})

extract.on('finish', function() \{
  // all entries read
\})

pack.pipe(extract)
\end{DoxyCode}


The tar archive is streamed sequentially, meaning you {\bfseries must} drain each entry\textquotesingle{}s stream as you get them or else the main extract stream will receive backpressure and stop reading.

\subsection*{Headers}

The header object using in {\ttfamily entry} should contain the following properties. Most of these values can be found by stat\textquotesingle{}ing a file.


\begin{DoxyCode}
\{
  name: 'path/to/this/entry.txt',
  size: 1314,        // entry size. defaults to 0
  mode: 0644,        // entry mode. defaults to to 0755 for dirs and 0644 otherwise
  mtime: new Date(), // last modified date for entry. defaults to now.
  type: 'file',      // type of entry. defaults to file. can be:
                     // file | link | symlink | directory | block-device
                     // character-device | fifo | contiguous-file
  linkname: 'path',  // linked file name
  uid: 0,            // uid of entry owner. defaults to 0
  gid: 0,            // gid of entry owner. defaults to 0
  uname: 'maf',      // uname of entry owner. defaults to null
  gname: 'staff',    // gname of entry owner. defaults to null
  devmajor: 0,       // device major version. defaults to 0
  devminor: 0        // device minor version. defaults to 0
\}
\end{DoxyCode}


\subsection*{Modifying existing tarballs}

Using tar-\/stream it is easy to rewrite paths / change modes etc in an existing tarball.


\begin{DoxyCode}
var extract = tar.extract()
var pack = tar.pack()
var path = require('path')

extract.on('entry', function(header, stream, callback) \{
  // let's prefix all names with 'tmp'
  header.name = path.join('tmp', header.name)
  // write the new entry to the pack stream
  stream.pipe(pack.entry(header, callback))
\})

extract.on('finish', function() \{
  // all entries done - lets finalize it
  pack.finalize()
\})

// pipe the old tarball to the extractor
oldTarballStream.pipe(extract)

// pipe the new tarball the another stream
pack.pipe(newTarballStream)
\end{DoxyCode}


\subsection*{Saving tarball to fs}


\begin{DoxyCode}
var fs = require('fs')
var tar = require('tar-stream')

var pack = tar.pack() // pack is a streams2 stream
var path = 'YourTarBall.tar'
var yourTarball = fs.createWriteStream(path)

// add a file called YourFile.txt with the content "Hello World!"
pack.entry(\{name: 'YourFile.txt'\}, 'Hello World!', function (err) \{
  if (err) throw err
  pack.finalize()
\})

// pipe the pack stream to your file
pack.pipe(yourTarball)

yourTarball.on('close', function () \{
  console.log(path + ' has been written')
  fs.stat(path, function(err, stats) \{
    if (err) throw err
    console.log(stats)
    console.log('Got file info successfully!')
  \})
\})
\end{DoxyCode}


\subsection*{Performance}

\href{https://github.com/mafintosh/tar-fs/blob/master/README.md#performance}{\tt See tar-\/fs for a performance comparison with node-\/tar}

\section*{License}

M\+IT 