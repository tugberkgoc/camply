\href{https://nodei.co/npm/mongodb/}{\tt } \href{https://nodei.co/npm/mongodb/}{\tt }

\href{http://travis-ci.org/mongodb/node-mongodb-native}{\tt } \href{https://coveralls.io/github/mongodb/node-mongodb-native?branch=2.1}{\tt } \href{https://gitter.im/mongodb/node-mongodb-native?utm_source=badge&utm_medium=badge&utm_campaign=pr-badge}{\tt }

\section*{Description}

The official \href{https://www.mongodb.com/}{\tt Mongo\+DB} driver for Node.\+js. Provides a high-\/level A\+PI on top of \href{https://www.npmjs.com/package/mongodb-core}{\tt mongodb-\/core} that is meant for end users.

{\bfseries N\+O\+TE\+: v3.\+x was recently released with breaking A\+PI changes. You can find a list of changes C\+H\+A\+N\+G\+E\+S\+\_\+3.0.\+0.\+md \char`\"{}here\char`\"{}.}

\subsection*{Mongo\+DB Node.\+JS Driver}

\tabulinesep=1mm
\begin{longtabu} spread 0pt [c]{*{2}{|X[-1]}|}
\hline
\rowcolor{\tableheadbgcolor}\textbf{ what  }&\textbf{ where   }\\\cline{1-2}
\endfirsthead
\hline
\endfoot
\hline
\rowcolor{\tableheadbgcolor}\textbf{ what  }&\textbf{ where   }\\\cline{1-2}
\endhead
documentation  &\href{http://mongodb.github.io/node-mongodb-native}{\tt http\+://mongodb.\+github.\+io/node-\/mongodb-\/native}   \\\cline{1-2}
api-\/doc  &\href{http://mongodb.github.io/node-mongodb-native/3.1/api}{\tt http\+://mongodb.\+github.\+io/node-\/mongodb-\/native/3.\+1/api}   \\\cline{1-2}
source  &\href{https://github.com/mongodb/node-mongodb-native}{\tt https\+://github.\+com/mongodb/node-\/mongodb-\/native}   \\\cline{1-2}
mongodb  &\href{http://www.mongodb.org}{\tt http\+://www.\+mongodb.\+org}   \\\cline{1-2}
\end{longtabu}


\subsubsection*{Bugs / Feature Requests}

Think you’ve found a bug? Want to see a new feature in {\ttfamily node-\/mongodb-\/native}? Please open a case in our issue management tool, J\+I\+RA\+:


\begin{DoxyItemize}
\item Create an account and login \href{https://jira.mongodb.org}{\tt jira.\+mongodb.\+org}.
\item Navigate to the N\+O\+DE project \href{https://jira.mongodb.org/browse/NODE}{\tt jira.\+mongodb.\+org/browse/\+N\+O\+DE}.
\item Click {\bfseries Create Issue} -\/ Please provide as much information as possible about the issue type and how to reproduce it.
\end{DoxyItemize}

Bug reports in J\+I\+RA for all driver projects (i.\+e. N\+O\+DE, P\+Y\+T\+H\+ON, C\+S\+H\+A\+RP, J\+A\+VA) and the Core Server (i.\+e. S\+E\+R\+V\+ER) project are {\bfseries public}.

\subsubsection*{Questions and Bug Reports}


\begin{DoxyItemize}
\item Mailing List\+: \href{https://groups.google.com/forum/#!forum/node-mongodb-native}{\tt groups.\+google.\+com/forum/\#!forum/node-\/mongodb-\/native}
\item J\+I\+RA\+: \href{http://jira.mongodb.org}{\tt jira.\+mongodb.\+org}
\end{DoxyItemize}

\subsubsection*{Change Log}

Change history can be found in \`{}\+H\+I\+S\+T\+O\+RY.md\`{}.

\section*{Installation}

The recommended way to get started using the Node.\+js 3.\+0 driver is by using the {\ttfamily npm} (Node Package Manager) to install the dependency in your project.

\subsection*{Mongo\+DB Driver}

Given that you have created your own project using {\ttfamily npm init} we install the Mongo\+DB driver and its dependencies by executing the following {\ttfamily npm} command.


\begin{DoxyCode}
npm install mongodb --save
\end{DoxyCode}


This will download the Mongo\+DB driver and add a dependency entry in your {\ttfamily package.\+json} file.

You can also use the \href{https://yarnpkg.com/en}{\tt Yarn} package manager.

\subsection*{Troubleshooting}

The Mongo\+DB driver depends on several other packages. These are\+:


\begin{DoxyItemize}
\item \href{https://github.com/mongodb-js/mongodb-core}{\tt mongodb-\/core}
\item \href{https://github.com/mongodb/js-bson}{\tt bson}
\item \href{https://github.com/mongodb-js/kerberos}{\tt kerberos}
\item \href{https://github.com/nodejs/node-gyp}{\tt node-\/gyp}
\end{DoxyItemize}

The {\ttfamily kerberos} package is a C++ extension that requires a build environment to be installed on your system. You must be able to build Node.\+js itself in order to compile and install the {\ttfamily kerberos} module. Furthermore, the {\ttfamily kerberos} module requires the M\+IT Kerberos package to correctly compile on U\+N\+IX operating systems. Consult your U\+N\+IX operation system package manager for what libraries to install.

{\bfseries Windows already contains the S\+S\+PI A\+PI used for Kerberos authentication. However, you will need to install a full compiler tool chain using Visual Studio C++ to correctly install the Kerberos extension.}

\subsubsection*{Diagnosing on U\+N\+IX}

If you don’t have the build-\/essentials, this module won’t build. In the case of Linux, you will need gcc, g++, Node.\+js with all the headers and Python. The easiest way to figure out what’s missing is by trying to build the Kerberos project. You can do this by performing the following steps.


\begin{DoxyCode}
git clone https://github.com/mongodb-js/kerberos
cd kerberos
npm install
\end{DoxyCode}


If all the steps complete, you have the right toolchain installed. If you get the error \char`\"{}node-\/gyp not found,\char`\"{} you need to install {\ttfamily node-\/gyp} globally\+:


\begin{DoxyCode}
npm install -g node-gyp
\end{DoxyCode}


If it correctly compiles and runs the tests you are golden. We can now try to install the {\ttfamily mongod} driver by performing the following command.


\begin{DoxyCode}
cd yourproject
npm install mongodb --save
\end{DoxyCode}


If it still fails the next step is to examine the npm log. Rerun the command but in this case in verbose mode.


\begin{DoxyCode}
npm --loglevel verbose install mongodb
\end{DoxyCode}


This will print out all the steps npm is performing while trying to install the module.

\subsubsection*{Diagnosing on Windows}

A compiler tool chain known to work for compiling {\ttfamily kerberos} on Windows is the following.


\begin{DoxyItemize}
\item Visual Studio C++ 2010 (do not use higher versions)
\item Windows 7 64bit S\+DK
\item Python 2.\+7 or higher
\end{DoxyItemize}

Open the Visual Studio command prompt. Ensure {\ttfamily node.\+exe} is in your path and install {\ttfamily node-\/gyp}.


\begin{DoxyCode}
npm install -g node-gyp
\end{DoxyCode}


Next, you will have to build the project manually to test it. Clone the repo, install dependencies and rebuild\+:


\begin{DoxyCode}
git clone https://github.com/christkv/kerberos.git
cd kerberos
npm install
node-gyp rebuild
\end{DoxyCode}


This should rebuild the driver successfully if you have everything set up correctly.

\subsubsection*{Other possible issues}

Your Python installation might be hosed making gyp break. Test your deployment environment first by trying to build Node.\+js itself on the server in question, as this should unearth any issues with broken packages (and there are a lot of broken packages out there).

Another tip is to ensure your user has write permission to wherever the Node.\+js modules are being installed.

\subsection*{Quick Start}

This guide will show you how to set up a simple application using Node.\+js and Mongo\+DB. Its scope is only how to set up the driver and perform the simple C\+R\+UD operations. For more in-\/depth coverage, see the tutorials.

\subsubsection*{Create the {\ttfamily package.\+json} file}

First, create a directory where your application will live.


\begin{DoxyCode}
mkdir myproject
cd myproject
\end{DoxyCode}


Enter the following command and answer the questions to create the initial structure for your new project\+:


\begin{DoxyCode}
npm init
\end{DoxyCode}


Next, install the driver dependency.


\begin{DoxyCode}
npm install mongodb --save
\end{DoxyCode}


You should see {\bfseries N\+PM} download a lot of files. Once it\textquotesingle{}s done you\textquotesingle{}ll find all the downloaded packages under the {\bfseries node\+\_\+modules} directory.

\subsubsection*{Start a Mongo\+DB Server}

For complete Mongo\+DB installation instructions, see \href{https://docs.mongodb.org/manual/installation/}{\tt the manual}.


\begin{DoxyEnumerate}
\item Download the right Mongo\+DB version from \href{https://www.mongodb.org/downloads}{\tt Mongo\+DB}
\item Create a database directory (in this case under $\ast$$\ast$/data$\ast$$\ast$).
\item Install and start a {\ttfamily mongod} process.
\end{DoxyEnumerate}


\begin{DoxyCode}
mongod --dbpath=/data
\end{DoxyCode}


You should see the {\bfseries mongod} process start up and print some status information.

\subsubsection*{Connect to Mongo\+DB}

Create a new {\bfseries app.\+js} file and add the following code to try out some basic C\+R\+UD operations using the Mongo\+DB driver.

Add code to connect to the server and the database {\bfseries myproject}\+:


\begin{DoxyCode}
const MongoClient = require('mongodb').MongoClient;
const assert = require('assert');

// Connection URL
const url = 'mongodb://localhost:27017';

// Database Name
const dbName = 'myproject';

// Use connect method to connect to the server
MongoClient.connect(url, function(err, client) \{
  assert.equal(null, err);
  console.log("Connected successfully to server");

  const db = client.db(dbName);

  client.close();
\});
\end{DoxyCode}


Run your app from the command line with\+:


\begin{DoxyCode}
node app.js
\end{DoxyCode}


The application should print {\bfseries Connected successfully to server} to the console.

\subsubsection*{Insert a Document}

Add to {\bfseries app.\+js} the following function which uses the {\bfseries insert\+Many} method to add three documents to the {\bfseries documents} collection.


\begin{DoxyCode}
const insertDocuments = function(db, callback) \{
  // Get the documents collection
  const collection = db.collection('documents');
  // Insert some documents
  collection.insertMany([
    \{a : 1\}, \{a : 2\}, \{a : 3\}
  ], function(err, result) \{
    assert.equal(err, null);
    assert.equal(3, result.result.n);
    assert.equal(3, result.ops.length);
    console.log("Inserted 3 documents into the collection");
    callback(result);
  \});
\}
\end{DoxyCode}


The {\bfseries insert} command returns an object with the following fields\+:


\begin{DoxyItemize}
\item {\bfseries result} Contains the result document from Mongo\+DB
\item {\bfseries ops} Contains the documents inserted with added $\ast$$\ast$\+\_\+id$\ast$$\ast$ fields
\item {\bfseries connection} Contains the connection used to perform the insert
\end{DoxyItemize}

Add the following code to call the {\bfseries insert\+Documents} function\+:


\begin{DoxyCode}
const MongoClient = require('mongodb').MongoClient;
const assert = require('assert');

// Connection URL
const url = 'mongodb://localhost:27017';

// Database Name
const dbName = 'myproject';

// Use connect method to connect to the server
MongoClient.connect(url, function(err, client) \{
  assert.equal(null, err);
  console.log("Connected successfully to server");

  const db = client.db(dbName);

  insertDocuments(db, function() \{
    client.close();
  \});
\});
\end{DoxyCode}


Run the updated {\bfseries app.\+js} file\+:


\begin{DoxyCode}
node app.js
\end{DoxyCode}


The operation returns the following output\+:


\begin{DoxyCode}
Connected successfully to server
Inserted 3 documents into the collection
\end{DoxyCode}


\subsubsection*{Find All Documents}

Add a query that returns all the documents.


\begin{DoxyCode}
const findDocuments = function(db, callback) \{
  // Get the documents collection
  const collection = db.collection('documents');
  // Find some documents
  collection.find(\{\}).toArray(function(err, docs) \{
    assert.equal(err, null);
    console.log("Found the following records");
    console.log(docs)
    callback(docs);
  \});
\}
\end{DoxyCode}


This query returns all the documents in the {\bfseries documents} collection. Add the {\bfseries find\+Document} method to the {\bfseries Mongo\+Client.\+connect} callback\+:


\begin{DoxyCode}
const MongoClient = require('mongodb').MongoClient;
const assert = require('assert');

// Connection URL
const url = 'mongodb://localhost:27017';

// Database Name
const dbName = 'myproject';

// Use connect method to connect to the server
MongoClient.connect(url, function(err, client) \{
  assert.equal(null, err);
  console.log("Connected correctly to server");

  const db = client.db(dbName);

  insertDocuments(db, function() \{
    findDocuments(db, function() \{
      client.close();
    \});
  \});
\});
\end{DoxyCode}


\subsubsection*{Find Documents with a Query Filter}

Add a query filter to find only documents which meet the query criteria.


\begin{DoxyCode}
const findDocuments = function(db, callback) \{
  // Get the documents collection
  const collection = db.collection('documents');
  // Find some documents
  collection.find(\{'a': 3\}).toArray(function(err, docs) \{
    assert.equal(err, null);
    console.log("Found the following records");
    console.log(docs);
    callback(docs);
  \});
\}
\end{DoxyCode}


Only the documents which match {\ttfamily \textquotesingle{}a\textquotesingle{} \+: 3} should be returned.

\subsubsection*{Update a document}

The following operation updates a document in the {\bfseries documents} collection.


\begin{DoxyCode}
const updateDocument = function(db, callback) \{
  // Get the documents collection
  const collection = db.collection('documents');
  // Update document where a is 2, set b equal to 1
  collection.updateOne(\{ a : 2 \}
    , \{ $set: \{ b : 1 \} \}, function(err, result) \{
    assert.equal(err, null);
    assert.equal(1, result.result.n);
    console.log("Updated the document with the field a equal to 2");
    callback(result);
  \});
\}
\end{DoxyCode}


The method updates the first document where the field {\bfseries a} is equal to {\bfseries 2} by adding a new field {\bfseries b} to the document set to {\bfseries 1}. Next, update the callback function from {\bfseries Mongo\+Client.\+connect} to include the update method.


\begin{DoxyCode}
const MongoClient = require('mongodb').MongoClient;
const assert = require('assert');

// Connection URL
const url = 'mongodb://localhost:27017';

// Database Name
const dbName = 'myproject';

// Use connect method to connect to the server
MongoClient.connect(url, function(err, client) \{
  assert.equal(null, err);
  console.log("Connected successfully to server");

  const db = client.db(dbName);

  insertDocuments(db, function() \{
    updateDocument(db, function() \{
      client.close();
    \});
  \});
\});
\end{DoxyCode}


\subsubsection*{Remove a document}

Remove the document where the field {\bfseries a} is equal to {\bfseries 3}.


\begin{DoxyCode}
const removeDocument = function(db, callback) \{
  // Get the documents collection
  const collection = db.collection('documents');
  // Delete document where a is 3
  collection.deleteOne(\{ a : 3 \}, function(err, result) \{
    assert.equal(err, null);
    assert.equal(1, result.result.n);
    console.log("Removed the document with the field a equal to 3");
    callback(result);
  \});
\}
\end{DoxyCode}


Add the new method to the {\bfseries Mongo\+Client.\+connect} callback function.


\begin{DoxyCode}
const MongoClient = require('mongodb').MongoClient;
const assert = require('assert');

// Connection URL
const url = 'mongodb://localhost:27017';

// Database Name
const dbName = 'myproject';

// Use connect method to connect to the server
MongoClient.connect(url, function(err, client) \{
  assert.equal(null, err);
  console.log("Connected successfully to server");

  const db = client.db(dbName);

  insertDocuments(db, function() \{
    updateDocument(db, function() \{
      removeDocument(db, function() \{
        client.close();
      \});
    \});
  \});
\});
\end{DoxyCode}


\subsubsection*{Index a Collection}

\href{https://docs.mongodb.org/manual/indexes/}{\tt Indexes} can improve your application\textquotesingle{}s performance. The following function creates an index on the {\bfseries a} field in the {\bfseries documents} collection.


\begin{DoxyCode}
const indexCollection = function(db, callback) \{
  db.collection('documents').createIndex(
    \{ "a": 1 \},
      null,
      function(err, results) \{
        console.log(results);
        callback();
    \}
  );
\};
\end{DoxyCode}


Add the {\ttfamily index\+Collection} method to your app\+:


\begin{DoxyCode}
const MongoClient = require('mongodb').MongoClient;
const assert = require('assert');

// Connection URL
const url = 'mongodb://localhost:27017';

const dbName = 'myproject';

// Use connect method to connect to the server
MongoClient.connect(url, function(err, client) \{
  assert.equal(null, err);
  console.log("Connected successfully to server");

  const db = client.db(dbName);

  insertDocuments(db, function() \{
    indexCollection(db, function() \{
      client.close();
    \});
  \});
\});
\end{DoxyCode}


For more detailed information, see the tutorials.

\subsection*{Next Steps}


\begin{DoxyItemize}
\item \href{http://mongodb.org}{\tt Mongo\+DB Documentation}
\item \href{http://learnmongodbthehardway.com}{\tt Read about Schemas}
\item \href{https://github.com/mongodb/node-mongodb-native}{\tt Star us on Git\+Hub}
\end{DoxyItemize}

\subsection*{License}

Apache 2.0

© 2009-\/2012 Christian Amor Kvalheim ~\newline
© 2012-\/present Mongo\+DB Contributors 