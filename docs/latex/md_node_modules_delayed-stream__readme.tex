Buffers events from a stream until you are ready to handle them.

\subsection*{Installation}


\begin{DoxyCode}
npm install delayed-stream
\end{DoxyCode}


\subsection*{Usage}

The following example shows how to write a http echo server that delays its response by 1000 ms.


\begin{DoxyCode}
var DelayedStream = require('delayed-stream');
var http = require('http');

http.createServer(function(req, res) \{
  var delayed = DelayedStream.create(req);

  setTimeout(function() \{
    res.writeHead(200);
    delayed.pipe(res);
  \}, 1000);
\});
\end{DoxyCode}


If you are not using {\ttfamily Stream\+::pipe}, you can also manually release the buffered events by calling {\ttfamily delayed\+Stream.\+resume()}\+:


\begin{DoxyCode}
var delayed = DelayedStream.create(req);

setTimeout(function() \{
  // Emit all buffered events and resume underlaying source
  delayed.resume();
\}, 1000);
\end{DoxyCode}


\subsection*{Implementation}

In order to use this meta stream properly, here are a few things you should know about the implementation.

\subsubsection*{Event Buffering / Proxying}

All events of the {\ttfamily source} stream are hijacked by overwriting the {\ttfamily source.\+emit} method. Until node implements a catch-\/all event listener, this is the only way.

However, delayed-\/stream still continues to emit all events it captures on the {\ttfamily source}, regardless of whether you have released the delayed stream yet or not.

Upon creation, delayed-\/stream captures all {\ttfamily source} events and stores them in an internal event buffer. Once {\ttfamily delayed\+Stream.\+release()} is called, all buffered events are emitted on the {\ttfamily delayed\+Stream}, and the event buffer is cleared. After that, delayed-\/stream merely acts as a proxy for the underlaying source.

\subsubsection*{Error handling}

Error events on {\ttfamily source} are buffered / proxied just like any other events. However, {\ttfamily delayed\+Stream.\+create} attaches a no-\/op `\textquotesingle{}error'{\ttfamily listener to the }source{\ttfamily . This way you only have to handle errors on the}delayed\+Stream\`{} object, rather than in two places.

\subsubsection*{Buffer limits}

delayed-\/stream provides a {\ttfamily max\+Data\+Size} property that can be used to limit the amount of data being buffered. In order to protect you from bad {\ttfamily source} streams that don\textquotesingle{}t react to {\ttfamily source.\+pause()}, this feature is enabled by default.

\subsection*{A\+PI}

\subsubsection*{Delayed\+Stream.\+create(source, \mbox{[}options\mbox{]})}

Returns a new {\ttfamily delayed\+Stream}. Available options are\+:


\begin{DoxyItemize}
\item {\ttfamily pause\+Stream}
\item {\ttfamily max\+Data\+Size}
\end{DoxyItemize}

The description for those properties can be found below.

\subsubsection*{delayed\+Stream.\+source}

The {\ttfamily source} stream managed by this object. This is useful if you are passing your {\ttfamily delayed\+Stream} around, and you still want to access properties on the {\ttfamily source} object.

\subsubsection*{delayed\+Stream.\+pause\+Stream = true}

Whether to pause the underlaying {\ttfamily source} when calling {\ttfamily Delayed\+Stream.\+create()}. Modifying this property afterwards has no effect.

\subsubsection*{delayed\+Stream.\+max\+Data\+Size = 1024 $\ast$ 1024}

The amount of data to buffer before emitting an {\ttfamily error}.

If the underlaying source is emitting {\ttfamily Buffer} objects, the {\ttfamily max\+Data\+Size} refers to bytes.

If the underlaying source is emitting Java\+Script strings, the size refers to characters.

If you know what you are doing, you can set this property to {\ttfamily Infinity} to disable this feature. You can also modify this property during runtime.

\subsubsection*{delayed\+Stream.\+data\+Size = 0}

The amount of data buffered so far.

\subsubsection*{delayed\+Stream.\+readable}

An E\+C\+M\+A5 getter that returns the value of {\ttfamily source.\+readable}.

\subsubsection*{delayed\+Stream.\+resume()}

If the {\ttfamily delayed\+Stream} has not been released so far, {\ttfamily delayed\+Stream.\+release()} is called.

In either case, {\ttfamily source.\+resume()} is called.

\subsubsection*{delayed\+Stream.\+pause()}

Calls {\ttfamily source.\+pause()}.

\subsubsection*{delayed\+Stream.\+pipe(dest)}

Calls {\ttfamily delayed\+Stream.\+resume()} and then proxies the arguments to {\ttfamily source.\+pipe}.

\subsubsection*{delayed\+Stream.\+release()}

Emits and clears all events that have been buffered up so far. This does not resume the underlaying source, use {\ttfamily delayed\+Stream.\+resume()} instead.

\subsection*{License}

delayed-\/stream is licensed under the M\+IT license. 