Returns {\ttfamily true} if the current environment is a Continuous Integration server.

Please \href{https://github.com/watson/is-ci/issues}{\tt open an issue} if your CI server isn\textquotesingle{}t properly detected \+:)

\href{https://www.npmjs.com/package/is-ci}{\tt } \href{https://travis-ci.org/watson/is-ci}{\tt } \href{https://github.com/feross/standard}{\tt }

\subsection*{Installation}


\begin{DoxyCode}
npm install is-ci --save
\end{DoxyCode}


\subsection*{Programmatic Usage}


\begin{DoxyCode}
const isCI = require('is-ci')

if (isCI) \{
  console.log('The code is running on a CI server')
\}
\end{DoxyCode}


\subsection*{C\+LI Usage}

For C\+LI usage you need to have the {\ttfamily is-\/ci} executable in your {\ttfamily P\+A\+TH}. There\textquotesingle{}s a few ways to do that\+:


\begin{DoxyItemize}
\item Either install the module globally using {\ttfamily npm install is-\/ci -\/g}
\item Or add the module as a dependency to your app in which case it can be used inside your package.\+json scripts as is
\item Or provide the full path to the executable, e.\+g. {\ttfamily ./node\+\_\+modules/.bin/is-\/ci}
\end{DoxyItemize}


\begin{DoxyCode}
is-ci && echo "This is a CI server"
\end{DoxyCode}


\subsection*{Supported CI tools}

Refer to \href{https://github.com/watson/ci-info#supported-ci-tools}{\tt ci-\/info} docs for all supported CI\textquotesingle{}s

\subsection*{License}

\mbox{[}M\+IT\mbox{]}(L\+I\+C\+E\+N\+SE) 