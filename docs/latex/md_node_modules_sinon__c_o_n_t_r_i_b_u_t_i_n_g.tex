There are several ways of contributing to Sinon.\+JS


\begin{DoxyItemize}
\item Look into \href{https://github.com/sinonjs/sinon/issues?q=is%3Aopen+is%3Aissue+label%3A%22Help+wanted%22}{\tt issues tagged {\ttfamily help-\/wanted}}
\item Help \href{https://github.com/sinonjs/sinon/tree/master/docs}{\tt improve the documentation} published at \href{https://sinonjs.org}{\tt the Sinon.\+JS website}. \href{https://github.com/sinonjs/sinon/issues?q=is%3Aopen+is%3Aissue+label%3ADocumentation}{\tt Documentation issues}.
\item Help someone understand and use Sinon.\+JS on \href{https://stackoverflow.com/questions/tagged/sinon}{\tt Stack Overflow}
\item Report an issue, please read instructions below
\item Help with triaging the \href{https://github.com/sinonjs/sinon/issues}{\tt issues}. The clearer they are, the more likely they are to be fixed soon.
\item Contribute to the code base.
\end{DoxyItemize}

\subsection*{Reporting an issue}

To save everyone time and make it much more likely for your issue to be understood, worked on and resolved quickly, it would help if you\textquotesingle{}re mindful of \href{http://www.chiark.greenend.org.uk/~sgtatham/bugs.html}{\tt How to Report Bugs Effectively} when pressing the \char`\"{}\+Submit new issue\char`\"{} button.

As a minimum, please report the following\+:


\begin{DoxyItemize}
\item Which environment are you using? Browser? Node? Which version(s)?
\item Which version of Sinon\+JS?
\item How are you loading Sinon\+JS?
\item What other libraries are you using?
\item What you expected to happen
\item What actually happens
\item Describe {\bfseries with code} how to reproduce the faulty behaviour
\end{DoxyItemize}

See \href{https://github.com/sinonjs/sinon/blob/master/.github/}{\tt our issue template} for all details.

\subsection*{Contributing to the code base}

Pick \href{https://github.com/sinonjs/sinon/issues}{\tt an issue} to fix, or pitch new features. To avoid wasting your time, please ask for feedback on feature suggestions with \href{https://github.com/sinonjs/sinon/issues/new}{\tt an issue}.

\subsubsection*{Making a pull request}

Please try to \href{http://chris.beams.io/posts/git-commit/}{\tt write great commit messages}.

There are numerous benefits to great commit messages


\begin{DoxyItemize}
\item They allow Sinon.\+JS users to easily understand the consequences of updating to a newer version
\item They help contributors understand what is going on with the codebase, allowing features and fixes to be developed faster
\item They save maintainers time when compiling the changelog for a new release
\end{DoxyItemize}

If you\textquotesingle{}re already a few commits in by the time you read this, you can still \href{https://help.github.com/articles/changing-a-commit-message/}{\tt change your commit messages}.

Also, before making your pull request, consider if your commits make sense on their own (and potentially should be multiple pull requests) or if they can be squashed down to one commit (with a great message). There are no hard and fast rules about this, but being mindful of your readers greatly help you author good commits.

\subsubsection*{Use Editor\+Config}

To save everyone some time, please use \href{http://editorconfig.org}{\tt Editor\+Config}, so your editor helps make sure we all use the same encoding, indentation, line endings, etc.

\subsubsection*{Installation}

The Sinon.\+JS developer environment requires Node/\+N\+PM. Please make sure you have Node installed, and install Sinon\textquotesingle{}s dependencies\+: \begin{DoxyVerb}$ npm install
\end{DoxyVerb}


This will also install a pre-\/commit hook, that runs style validation on staged files.

\subsubsection*{Compatibility}

\paragraph*{E\+S5.\+1}

Sinon\textquotesingle{}s source is written as \href{http://www.ecma-international.org/ecma-262/5.1/}{\tt E\+S5.\+1} and requires no transpiler or polyfills.

Sinon.\+JS uses feature detection to support \href{http://www.ecma-international.org/ecma-262/6.0/}{\tt E\+S6} features, but does not rely on any of the new syntax introduced in \href{http://www.ecma-international.org/ecma-262/6.0/}{\tt E\+S6} and remains compatible with \href{http://www.ecma-international.org/ecma-262/5.1/}{\tt E\+S5.\+1} runtimes.

\paragraph*{Runtimes}

Sinon.\+JS aims at supporting the following runtimes\+:


\begin{DoxyItemize}
\item Firefox 45+
\item Chrome 48+
\item Internet Explorer 11+
\item Edge 14+
\item Safari 9+
\item Node L\+TS versions
\end{DoxyItemize}

\subsubsection*{Linting and style}

Sinon.\+JS uses \href{http://eslint.org}{\tt E\+S\+Lint} to keep the codebase free of lint, and uses \href{https://prettier.io}{\tt Prettier} to keep consistent style.

If you are contributing to a Sinon project, you\textquotesingle{}ll probably want to configure your editors (\href{https://eslint.org/docs/user-guide/integrations#editors}{\tt E\+S\+Lint}, \href{https://prettier.io/docs/en/editors.html}{\tt Prettier}) to make editing code a more enjoyable experience.

The E\+S\+Lint verification (which includes Prettier) will be run before unit tests in the CI environment. The build will fail if the source code does not pass the style check.

You can run the linter locally\+:


\begin{DoxyCode}
$ npm run lint
\end{DoxyCode}


You can fix a lot lint and style violations automatically\+:


\begin{DoxyCode}
$ npm run lint -- --fix
\end{DoxyCode}


To ensure consistent reporting of lint warnings, you should use the same versions of E\+S\+Lint and Prettier as defined in {\ttfamily package.\+json} (which is what the CI servers use).

\subsubsection*{Run the tests}

Following command runs unit tests in Phantom\+JS, Node and Web\+Worker \begin{DoxyVerb}$ npm test
\end{DoxyVerb}


\subparagraph*{Testing in development}

Sinon.\+JS uses \href{https://mochajs.org/}{\tt Mocha}, please read those docs if you\textquotesingle{}re unfamiliar with it.

If you\textquotesingle{}re doing more than a one line edit, you\textquotesingle{}ll want to have finer control and less restarting of the Mocha

To start tests in dev mode run \begin{DoxyVerb}$ npm run test-dev
\end{DoxyVerb}


Dev mode features\+:
\begin{DoxyItemize}
\item \href{https://mochajs.org/#w---watch}{\tt watching related files} to restart tests once changes are made
\item using \href{https://mochajs.org/#min}{\tt Min reporter}, which cleans the console each time tests run, so test results are always on top
\end{DoxyItemize}

Note that in dev mode tests run only in Node. Before creating your PR please ensure tests are passing in Phantom and Web\+Worker as well. To check this please use \href{#run-the-tests}{\tt Run the tests} instructions.

\subsubsection*{Compiling a built version}

Build requires Node. Under the hood \href{http://browserify.org/}{\tt Browserify} is used.

To build simply run \begin{DoxyVerb}$ node build.js
\end{DoxyVerb}
 