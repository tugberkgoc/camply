The Node.\+js Streams is jointly governed by a Working Group (WG) that is responsible for high-\/level guidance of the project.

The WG has final authority over this project including\+:


\begin{DoxyItemize}
\item Technical direction
\item Project governance and process (including this policy)
\item Contribution policy
\item Git\+Hub repository hosting
\item Conduct guidelines
\item Maintaining the list of additional Collaborators
\end{DoxyItemize}

For the current list of WG members, see the project \href{./README.md#current-project-team-members}{\tt R\+E\+A\+D\+M\+E.\+md}.

\subsubsection*{Collaborators}

The readable-\/stream Git\+Hub repository is maintained by the WG and additional Collaborators who are added by the WG on an ongoing basis.

Individuals making significant and valuable contributions are made Collaborators and given commit-\/access to the project. These individuals are identified by the WG and their addition as Collaborators is discussed during the WG meeting.

{\itshape Note\+:} If you make a significant contribution and are not considered for commit-\/access log an issue or contact a WG member directly and it will be brought up in the next WG meeting.

Modifications of the contents of the readable-\/stream repository are made on a collaborative basis. Anybody with a Git\+Hub account may propose a modification via pull request and it will be considered by the project Collaborators. All pull requests must be reviewed and accepted by a Collaborator with sufficient expertise who is able to take full responsibility for the change. In the case of pull requests proposed by an existing Collaborator, an additional Collaborator is required for sign-\/off. Consensus should be sought if additional Collaborators participate and there is disagreement around a particular modification. See {\itshape Consensus Seeking Process} below for further detail on the consensus model used for governance.

Collaborators may opt to elevate significant or controversial modifications, or modifications that have not found consensus to the WG for discussion by assigning the {\itshape {\bfseries W\+G-\/agenda}} tag to a pull request or issue. The WG should serve as the final arbiter where required.

For the current list of Collaborators, see the project \href{./README.md#members}{\tt R\+E\+A\+D\+M\+E.\+md}.

\subsubsection*{WG Membership}

WG seats are not time-\/limited. There is no fixed size of the WG. However, the expected target is between 6 and 12, to ensure adequate coverage of important areas of expertise, balanced with the ability to make decisions efficiently.

There is no specific set of requirements or qualifications for WG membership beyond these rules.

The WG may add additional members to the WG by unanimous consensus.

A WG member may be removed from the WG by voluntary resignation, or by unanimous consensus of all other WG members.

Changes to WG membership should be posted in the agenda, and may be suggested as any other agenda item (see \char`\"{}\+W\+G Meetings\char`\"{} below).

If an addition or removal is proposed during a meeting, and the full WG is not in attendance to participate, then the addition or removal is added to the agenda for the subsequent meeting. This is to ensure that all members are given the opportunity to participate in all membership decisions. If a WG member is unable to attend a meeting where a planned membership decision is being made, then their consent is assumed.

No more than 1/3 of the WG members may be affiliated with the same employer. If removal or resignation of a WG member, or a change of employment by a WG member, creates a situation where more than 1/3 of the WG membership shares an employer, then the situation must be immediately remedied by the resignation or removal of one or more WG members affiliated with the over-\/represented employer(s).

\subsubsection*{WG Meetings}

The WG meets occasionally on a Google Hangout On Air. A designated moderator approved by the WG runs the meeting. Each meeting should be published to You\+Tube.

Items are added to the WG agenda that are considered contentious or are modifications of governance, contribution policy, WG membership, or release process.

The intention of the agenda is not to approve or review all patches; that should happen continuously on Git\+Hub and be handled by the larger group of Collaborators.

Any community member or contributor can ask that something be added to the next meeting\textquotesingle{}s agenda by logging a Git\+Hub Issue. Any Collaborator, WG member or the moderator can add the item to the agenda by adding the {\itshape {\bfseries W\+G-\/agenda}} tag to the issue.

Prior to each WG meeting the moderator will share the Agenda with members of the WG. WG members can add any items they like to the agenda at the beginning of each meeting. The moderator and the WG cannot veto or remove items.

The WG may invite persons or representatives from certain projects to participate in a non-\/voting capacity.

The moderator is responsible for summarizing the discussion of each agenda item and sends it as a pull request after the meeting.

\subsubsection*{Consensus Seeking Process}

The WG follows a \href{http://en.wikipedia.org/wiki/Consensus-seeking_decision-making}{\tt Consensus Seeking} decision-\/making model.

When an agenda item has appeared to reach a consensus the moderator will ask \char`\"{}\+Does anyone object?\char`\"{} as a final call for dissent from the consensus.

If an agenda item cannot reach a consensus a WG member can call for either a closing vote or a vote to table the issue to the next meeting. The call for a vote must be seconded by a majority of the WG or else the discussion will continue. Simple majority wins.

Note that changes to WG membership require a majority consensus. See \char`\"{}\+W\+G Membership\char`\"{} above. 