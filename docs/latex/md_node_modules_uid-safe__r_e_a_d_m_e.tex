\href{https://npmjs.org/package/uid-safe}{\tt } \href{https://npmjs.org/package/uid-safe}{\tt } \href{https://nodejs.org/en/download/}{\tt } \href{https://travis-ci.org/crypto-utils/uid-safe}{\tt } \href{https://coveralls.io/r/crypto-utils/uid-safe?branch=master}{\tt }

U\+RL and cookie safe U\+I\+Ds

Create cryptographically secure U\+I\+Ds safe for both cookie and U\+RL usage. This is in contrast to modules such as \href{https://www.npmjs.com/package/rand-token}{\tt rand-\/token} and \href{https://www.npmjs.com/package/uid2}{\tt uid2} whose U\+I\+Ds are actually skewed due to the use of {\ttfamily \%} and unnecessarily truncate the U\+ID. Use this if you could still use U\+I\+Ds with {\ttfamily -\/} and {\ttfamily \+\_\+} in them.

\subsection*{Installation}


\begin{DoxyCode}
$ npm install uid-safe
\end{DoxyCode}


\subsection*{A\+PI}


\begin{DoxyCode}
var uid = require('uid-safe')
\end{DoxyCode}


\subsubsection*{uid(byte\+Length, callback)}

Asynchronously create a U\+ID with a specific byte length. Because {\ttfamily base64} encoding is used underneath, this is not the string length. For example, to create a U\+ID of length 24, you want a byte length of 18.


\begin{DoxyCode}
uid(18, function (err, string) \{
  if (err) throw err
  // do something with the string
\})
\end{DoxyCode}


\subsubsection*{uid(byte\+Length)}

Asynchronously create a U\+ID with a specific byte length and return a {\ttfamily Promise}.

{\bfseries Note}\+: To use promises in Node.\+js {\itshape prior to 0.\+12}, promises must be \char`\"{}polyfilled\char`\"{} using `global.\+Promise = require(\textquotesingle{}bluebird')\`{}.


\begin{DoxyCode}
uid(18).then(function (string) \{
  // do something with the string
\})
\end{DoxyCode}


\subsubsection*{uid.\+sync(byte\+Length)}

A synchronous version of above.


\begin{DoxyCode}
var string = uid.sync(18)
\end{DoxyCode}


\subsection*{License}

\mbox{[}M\+IT\mbox{]}(L\+I\+C\+E\+N\+SE) 