\href{https://npmjs.org/package/content-disposition}{\tt } \href{https://npmjs.org/package/content-disposition}{\tt } \href{https://nodejs.org/en/download}{\tt } \href{https://travis-ci.org/jshttp/content-disposition}{\tt } \href{https://coveralls.io/r/jshttp/content-disposition?branch=master}{\tt }

Create and parse H\+T\+TP {\ttfamily Content-\/\+Disposition} header

\subsection*{Installation}


\begin{DoxyCode}
$ npm install content-disposition
\end{DoxyCode}


\subsection*{A\+PI}


\begin{DoxyCode}
var contentDisposition = require('content-disposition')
\end{DoxyCode}


\subsubsection*{content\+Disposition(filename, options)}

Create an attachment {\ttfamily Content-\/\+Disposition} header value using the given file name, if supplied. The {\ttfamily filename} is optional and if no file name is desired, but you want to specify {\ttfamily options}, set {\ttfamily filename} to {\ttfamily undefined}.


\begin{DoxyCode}
res.setHeader('Content-Disposition', contentDisposition('∫ maths.pdf'))
\end{DoxyCode}


{\bfseries note} H\+T\+TP headers are of the I\+S\+O-\/8859-\/1 character set. If you are writing this header through a means different from {\ttfamily set\+Header} in Node.\+js, you\textquotesingle{}ll want to specify the `\textquotesingle{}binary'\`{} encoding in Node.\+js.

\paragraph*{Options}

{\ttfamily content\+Disposition} accepts these properties in the options object.

\subparagraph*{fallback}

If the {\ttfamily filename} option is outside I\+S\+O-\/8859-\/1, then the file name is actually stored in a supplemental field for clients that support Unicode file names and a I\+S\+O-\/8859-\/1 version of the file name is automatically generated.

This specifies the I\+S\+O-\/8859-\/1 file name to override the automatic generation or disables the generation all together, defaults to {\ttfamily true}.


\begin{DoxyItemize}
\item A string will specify the I\+S\+O-\/8859-\/1 file name to use in place of automatic generation.
\item {\ttfamily false} will disable including a I\+S\+O-\/8859-\/1 file name and only include the Unicode version (unless the file name is already I\+S\+O-\/8859-\/1).
\item {\ttfamily true} will enable automatic generation if the file name is outside I\+S\+O-\/8859-\/1.
\end{DoxyItemize}

If the {\ttfamily filename} option is I\+S\+O-\/8859-\/1 and this option is specified and has a different value, then the {\ttfamily filename} option is encoded in the extended field and this set as the fallback field, even though they are both I\+S\+O-\/8859-\/1.

\subparagraph*{type}

Specifies the disposition type, defaults to {\ttfamily \char`\"{}attachment\char`\"{}}. This can also be {\ttfamily \char`\"{}inline\char`\"{}}, or any other value (all values except inline are treated like {\ttfamily attachment}, but can convey additional information if both parties agree to it). The type is normalized to lower-\/case.

\subsubsection*{content\+Disposition.\+parse(string)}


\begin{DoxyCode}
var disposition = contentDisposition.parse('attachment; filename="EURO rates.txt";
       filename*=UTF-8\(\backslash\)'\(\backslash\)'%e2%82%ac%20rates.txt');
\end{DoxyCode}


Parse a {\ttfamily Content-\/\+Disposition} header string. This automatically handles extended (\char`\"{}\+Unicode\char`\"{}) parameters by decoding them and providing them under the standard parameter name. This will return an object with the following properties (examples are shown for the string `\textquotesingle{}attachment; filename=\char`\"{}\+E\+U\+R\+O rates.\+txt\char`\"{}; filename$\ast$=U\+T\+F-\/8\&rsquo;\textbackslash{}\textquotesingle{}e2\%82ac\%20rates.\+txt\textquotesingle{}\`{})\+:


\begin{DoxyItemize}
\item {\ttfamily type}\+: The disposition type (always lower case). Example\+: `\textquotesingle{}attachment'\`{}
\item {\ttfamily parameters}\+: An object of the parameters in the disposition (name of parameter always lower case and extended versions replace non-\/extended versions). Example\+: {\ttfamily \{filename\+: \char`\"{}€ rates.\+txt\char`\"{}\}}
\end{DoxyItemize}

\subsection*{Examples}

\subsubsection*{Send a file for download}


\begin{DoxyCode}
var contentDisposition = require('content-disposition')
var destroy = require('destroy')
var http = require('http')
var onFinished = require('on-finished')

var filePath = '/path/to/public/plans.pdf'

http.createServer(function onRequest(req, res) \{
  // set headers
  res.setHeader('Content-Type', 'application/pdf')
  res.setHeader('Content-Disposition', contentDisposition(filePath))

  // send file
  var stream = fs.createReadStream(filePath)
  stream.pipe(res)
  onFinished(res, function (err) \{
    destroy(stream)
  \})
\})
\end{DoxyCode}


\subsection*{Testing}


\begin{DoxyCode}
$ npm test
\end{DoxyCode}


\subsection*{References}


\begin{DoxyItemize}
\item \href{https://tools.ietf.org/html/rfc2616}{\tt R\+FC 2616\+: Hypertext Transfer Protocol -- H\+T\+T\+P/1.\+1}
\item \href{https://tools.ietf.org/html/rfc5987}{\tt R\+FC 5987\+: Character Set and Language Encoding for Hypertext Transfer Protocol (H\+T\+TP) Header Field Parameters}
\item \href{https://tools.ietf.org/html/rfc6266}{\tt R\+FC 6266\+: Use of the Content-\/\+Disposition Header Field in the Hypertext Transfer Protocol (H\+T\+TP)}
\item \href{http://greenbytes.de/tech/tc2231/}{\tt Test Cases for H\+T\+TP Content-\/\+Disposition header field (R\+FC 6266) and the Encodings defined in R\+F\+Cs 2047, 2231 and 5987}
\end{DoxyItemize}

\subsection*{License}

\mbox{[}M\+IT\mbox{]}(L\+I\+C\+E\+N\+SE) 