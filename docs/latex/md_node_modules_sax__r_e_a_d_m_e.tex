A sax-\/style parser for X\+ML and H\+T\+ML.

Designed with \href{http://nodejs.org/}{\tt node} in mind, but should work fine in the browser or other Common\+JS implementations.

\subsection*{What This Is}


\begin{DoxyItemize}
\item A very simple tool to parse through an X\+ML string.
\item A stepping stone to a streaming H\+T\+ML parser.
\item A handy way to deal with R\+SS and other mostly-\/ok-\/but-\/kinda-\/broken X\+ML docs.
\end{DoxyItemize}

\subsection*{What This Is (probably) Not}


\begin{DoxyItemize}
\item An H\+T\+ML Parser -\/ That\textquotesingle{}s a fine goal, but this isn\textquotesingle{}t it. It\textquotesingle{}s just X\+ML.
\item A D\+OM Builder -\/ You can use it to build an object model out of X\+ML, but it doesn\textquotesingle{}t do that out of the box.
\item X\+S\+LT -\/ No D\+OM = no querying.
\item 100\% Compliant with (some other S\+AX implementation) -\/ Most S\+AX implementations are in Java and do a lot more than this does.
\item An X\+ML Validator -\/ It does a little validation when in strict mode, but not much.
\item A Schema-\/\+Aware X\+SD Thing -\/ Schemas are an exercise in fetishistic masochism.
\item A D\+T\+D-\/aware Thing -\/ Fetching D\+T\+Ds is a much bigger job.
\end{DoxyItemize}

\subsection*{Regarding {\ttfamily $<$!\+D\+O\+C\+T\+Y\+PE}s and {\ttfamily $<$!\+E\+N\+T\+I\+TY}s}

The parser will handle the basic X\+ML entities in text nodes and attribute values\+: {\ttfamily \& $<$ $>$ \textquotesingle{} "}. It\textquotesingle{}s possible to define additional entities in X\+ML by putting them in the D\+TD. This parser doesn\textquotesingle{}t do anything with that. If you want to listen to the {\ttfamily ondoctype} event, and then fetch the doctypes, and read the entities and add them to {\ttfamily parser.\+E\+N\+T\+I\+T\+I\+ES}, then be my guest.

Unknown entities will fail in strict mode, and in loose mode, will pass through unmolested.

\subsection*{Usage}


\begin{DoxyCode}
var sax = require("./lib/sax"),
  strict = true, // set to false for html-mode
  parser = sax.parser(strict);

parser.onerror = function (e) \{
  // an error happened.
\};
parser.ontext = function (t) \{
  // got some text.  t is the string of text.
\};
parser.onopentag = function (node) \{
  // opened a tag.  node has "name" and "attributes"
\};
parser.onattribute = function (attr) \{
  // an attribute.  attr has "name" and "value"
\};
parser.onend = function () \{
  // parser stream is done, and ready to have more stuff written to it.
\};

parser.write('<xml>Hello, <who name="world">world</who>!</xml>').close();

// stream usage
// takes the same options as the parser
var saxStream = require("sax").createStream(strict, options)
saxStream.on("error", function (e) \{
  // unhandled errors will throw, since this is a proper node
  // event emitter.
  console.error("error!", e)
  // clear the error
  this.\_parser.error = null
  this.\_parser.resume()
\})
saxStream.on("opentag", function (node) \{
  // same object as above
\})
// pipe is supported, and it's readable/writable
// same chunks coming in also go out.
fs.createReadStream("file.xml")
  .pipe(saxStream)
  .pipe(fs.createWriteStream("file-copy.xml"))
\end{DoxyCode}


\subsection*{Arguments}

Pass the following arguments to the parser function. All are optional.

{\ttfamily strict} -\/ Boolean. Whether or not to be a jerk. Default\+: {\ttfamily false}.

{\ttfamily opt} -\/ Object bag of settings regarding string formatting. All default to {\ttfamily false}.

Settings supported\+:


\begin{DoxyItemize}
\item {\ttfamily trim} -\/ Boolean. Whether or not to trim text and comment nodes.
\item {\ttfamily normalize} -\/ Boolean. If true, then turn any whitespace into a single space.
\item {\ttfamily lowercase} -\/ Boolean. If true, then lowercase tag names and attribute names in loose mode, rather than uppercasing them.
\item {\ttfamily xmlns} -\/ Boolean. If true, then namespaces are supported.
\item {\ttfamily position} -\/ Boolean. If false, then don\textquotesingle{}t track line/col/position.
\item {\ttfamily strict\+Entities} -\/ Boolean. If true, only parse \href{http://www.w3.org/TR/REC-xml/#sec-predefined-ent}{\tt predefined X\+ML entities} ({\ttfamily \&}, {\ttfamily \textquotesingle{}}, {\ttfamily $>$}, {\ttfamily $<$}, and {\ttfamily "})
\end{DoxyItemize}

\subsection*{Methods}

{\ttfamily write} -\/ Write bytes onto the stream. You don\textquotesingle{}t have to do this all at once. You can keep writing as much as you want.

{\ttfamily close} -\/ Close the stream. Once closed, no more data may be written until it is done processing the buffer, which is signaled by the {\ttfamily end} event.

{\ttfamily resume} -\/ To gracefully handle errors, assign a listener to the {\ttfamily error} event. Then, when the error is taken care of, you can call {\ttfamily resume} to continue parsing. Otherwise, the parser will not continue while in an error state.

\subsection*{Members}

At all times, the parser object will have the following members\+:

{\ttfamily line}, {\ttfamily column}, {\ttfamily position} -\/ Indications of the position in the X\+ML document where the parser currently is looking.

{\ttfamily start\+Tag\+Position} -\/ Indicates the position where the current tag starts.

{\ttfamily closed} -\/ Boolean indicating whether or not the parser can be written to. If it\textquotesingle{}s {\ttfamily true}, then wait for the {\ttfamily ready} event to write again.

{\ttfamily strict} -\/ Boolean indicating whether or not the parser is a jerk.

{\ttfamily opt} -\/ Any options passed into the constructor.

{\ttfamily tag} -\/ The current tag being dealt with.

And a bunch of other stuff that you probably shouldn\textquotesingle{}t touch.

\subsection*{Events}

All events emit with a single argument. To listen to an event, assign a function to {\ttfamily on$<$eventname$>$}. Functions get executed in the this-\/context of the parser object. The list of supported events are also in the exported {\ttfamily E\+V\+E\+N\+TS} array.

When using the stream interface, assign handlers using the Event\+Emitter {\ttfamily on} function in the normal fashion.

{\ttfamily error} -\/ Indication that something bad happened. The error will be hanging out on {\ttfamily parser.\+error}, and must be deleted before parsing can continue. By listening to this event, you can keep an eye on that kind of stuff. Note\+: this happens {\itshape much} more in strict mode. Argument\+: instance of {\ttfamily Error}.

{\ttfamily text} -\/ Text node. Argument\+: string of text.

{\ttfamily doctype} -\/ The {\ttfamily $<$!\+D\+O\+C\+T\+Y\+PE} declaration. Argument\+: doctype string.

{\ttfamily processinginstruction} -\/ Stuff like {\ttfamily $<$?xml foo=\char`\"{}blerg\char`\"{} ?$>$}. Argument\+: object with {\ttfamily name} and {\ttfamily body} members. Attributes are not parsed, as processing instructions have implementation dependent semantics.

{\ttfamily sgmldeclaration} -\/ Random S\+G\+ML declarations. Stuff like {\ttfamily $<$!\+E\+N\+T\+I\+TY p$>$} would trigger this kind of event. This is a weird thing to support, so it might go away at some point. S\+AX isn\textquotesingle{}t intended to be used to parse S\+G\+ML, after all.

{\ttfamily opentagstart} -\/ Emitted immediately when the tag name is available, but before any attributes are encountered. Argument\+: object with a {\ttfamily name} field and an empty {\ttfamily attributes} set. Note that this is the same object that will later be emitted in the {\ttfamily opentag} event.

{\ttfamily opentag} -\/ An opening tag. Argument\+: object with {\ttfamily name} and {\ttfamily attributes}. In non-\/strict mode, tag names are uppercased, unless the {\ttfamily lowercase} option is set. If the {\ttfamily xmlns} option is set, then it will contain namespace binding information on the {\ttfamily ns} member, and will have a {\ttfamily local}, {\ttfamily prefix}, and {\ttfamily uri} member.

{\ttfamily closetag} -\/ A closing tag. In loose mode, tags are auto-\/closed if their parent closes. In strict mode, well-\/formedness is enforced. Note that self-\/closing tags will have {\ttfamily close\+Tag} emitted immediately after {\ttfamily open\+Tag}. Argument\+: tag name.

{\ttfamily attribute} -\/ An attribute node. Argument\+: object with {\ttfamily name} and {\ttfamily value}. In non-\/strict mode, attribute names are uppercased, unless the {\ttfamily lowercase} option is set. If the {\ttfamily xmlns} option is set, it will also contains namespace information.

{\ttfamily comment} -\/ A comment node. Argument\+: the string of the comment.

{\ttfamily opencdata} -\/ The opening tag of a {\ttfamily $<$!\mbox{[}C\+D\+A\+TA\mbox{[}} block.

{\ttfamily cdata} -\/ The text of a {\ttfamily $<$!\mbox{[}C\+D\+A\+TA\mbox{[}} block. Since {\ttfamily $<$!\mbox{[}C\+D\+A\+TA\mbox{[}} blocks can get quite large, this event may fire multiple times for a single block, if it is broken up into multiple {\ttfamily write()}s. Argument\+: the string of random character data.

{\ttfamily closecdata} -\/ The closing tag ({\ttfamily \mbox{]}\mbox{]}$>$}) of a {\ttfamily $<$!\mbox{[}C\+D\+A\+TA\mbox{[}} block.

{\ttfamily opennamespace} -\/ If the {\ttfamily xmlns} option is set, then this event will signal the start of a new namespace binding.

{\ttfamily closenamespace} -\/ If the {\ttfamily xmlns} option is set, then this event will signal the end of a namespace binding.

{\ttfamily end} -\/ Indication that the closed stream has ended.

{\ttfamily ready} -\/ Indication that the stream has reset, and is ready to be written to.

{\ttfamily noscript} -\/ In non-\/strict mode, {\ttfamily $<$script$>$} tags trigger a {\ttfamily \char`\"{}script\char`\"{}} event, and their contents are not checked for special xml characters. If you pass {\ttfamily noscript\+: true}, then this behavior is suppressed.

\subsection*{Reporting Problems}

It\textquotesingle{}s best to write a failing test if you find an issue. I will always accept pull requests with failing tests if they demonstrate intended behavior, but it is very hard to figure out what issue you\textquotesingle{}re describing without a test. Writing a test is also the best way for you yourself to figure out if you really understand the issue you think you have with sax-\/js. 