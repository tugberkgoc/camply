Like the unix {\ttfamily which} utility.

Finds the first instance of a specified executable in the P\+A\+TH environment variable. Does not cache the results, so {\ttfamily hash -\/r} is not needed when the P\+A\+TH changes.

\subsection*{U\+S\+A\+GE}


\begin{DoxyCode}
var which = require('which')

// async usage
which('node', function (er, resolvedPath) \{
  // er is returned if no "node" is found on the PATH
  // if it is found, then the absolute path to the exec is returned
\})

// sync usage
// throws if not found
var resolved = which.sync('node')

// if nothrow option is used, returns null if not found
resolved = which.sync('node', \{nothrow: true\})

// Pass options to override the PATH and PATHEXT environment vars.
which('node', \{ path: someOtherPath \}, function (er, resolved) \{
  if (er)
    throw er
  console.log('found at %j', resolved)
\})
\end{DoxyCode}


\subsection*{C\+LI U\+S\+A\+GE}

Same as the B\+SD {\ttfamily which(1)} binary.


\begin{DoxyCode}
usage: which [-as] program ...
\end{DoxyCode}


\subsection*{O\+P\+T\+I\+O\+NS}

You may pass an options object as the second argument.


\begin{DoxyItemize}
\item {\ttfamily path}\+: Use instead of the {\ttfamily P\+A\+TH} environment variable.
\item {\ttfamily path\+Ext}\+: Use instead of the {\ttfamily P\+A\+T\+H\+E\+XT} environment variable.
\item {\ttfamily all}\+: Return all matches, instead of just the first one. Note that this means the function returns an array of strings instead of a single string. 
\end{DoxyItemize}