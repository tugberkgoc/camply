{\bfseries Why use this?}

\begin{quote}
Split a string on a character except when the character is escaped. \end{quote}


Please consider following this project\textquotesingle{}s author, \href{https://github.com/jonschlinkert}{\tt Jon Schlinkert}, and consider starring the project to show your \+:heart\+: and support.

\subsection*{Install}

Install with \href{https://www.npmjs.com/}{\tt npm}\+:


\begin{DoxyCode}
$ npm install --save split-string
\end{DoxyCode}


$<$details$>$

~\newline


Although it\textquotesingle{}s easy to split on a string\+:


\begin{DoxyCode}
console.log('a.b.c'.split('.'));
//=> ['a', 'b', 'c']
\end{DoxyCode}


It\textquotesingle{}s more challenging to split a string whilst respecting escaped or quoted characters.

{\bfseries Bad}


\begin{DoxyCode}
console.log('a\(\backslash\)\(\backslash\).b.c'.split('.'));
//=> ['a\(\backslash\)\(\backslash\)', 'b', 'c']

console.log('"a.b.c".d'.split('.'));
//=> ['"a', 'b', 'c"', 'd']
\end{DoxyCode}


{\bfseries Good}


\begin{DoxyCode}
var split = require('split-string');
console.log(split('a\(\backslash\)\(\backslash\).b.c'));
//=> ['a.b', 'c']

console.log(split('"a.b.c".d'));
//=> ['a.b.c', 'd']
\end{DoxyCode}


See the \href{#options}{\tt options} to learn how to choose the separator or retain quotes or escaping.

~\newline


$<$/details$>$

\subsection*{Usage}


\begin{DoxyCode}
var split = require('split-string');

split('a.b.c');
//=> ['a', 'b', 'c']

// respects escaped characters
split('a.b.c\(\backslash\)\(\backslash\).d');
//=> ['a', 'b', 'c.d']

// respects double-quoted strings
split('a."b.c.d".e');
//=> ['a', 'b.c.d', 'e']
\end{DoxyCode}


{\bfseries Brackets}

Also respects brackets \href{#optionsbrackets}{\tt unless disabled}\+:


\begin{DoxyCode}
split('a (b c d) e', ' ');
//=> ['a', '(b c d)', 'e']
\end{DoxyCode}


\subsection*{Options}

\subsubsection*{options.\+brackets}

{\bfseries Type}\+: {\ttfamily object$\vert$boolean}

{\bfseries Default}\+: {\ttfamily undefined}

{\bfseries Description}

If enabled, split-\/string will not split inside brackets. The following brackets types are supported when {\ttfamily options.\+brackets} is {\ttfamily true},


\begin{DoxyCode}
\{
  '<': '>',
  '(': ')',
  '[': ']',
  '\{': '\}'
\}
\end{DoxyCode}


Or, if object of brackets must be passed, each property on the object must be a bracket type, where the property key is the opening delimiter and property value is the closing delimiter.

{\bfseries Examples}


\begin{DoxyCode}
// no bracket support by default
split('a.\{b.c\}');
//=> [ 'a', '\{b', 'c\}' ]

// support all basic bracket types: "<>\{\}[]()"
split('a.\{b.c\}', \{brackets: true\});
//=> [ 'a', '\{b.c\}' ]

// also supports nested brackets 
split('a.\{b.\{c.d\}.e\}.f', \{brackets: true\});
//=> [ 'a', '\{b.\{c.d\}.e\}', 'f' ]

// support only the specified brackets
split('[a.b].(c.d)', \{brackets: \{'[': ']'\}\});
//=> [ '[a.b]', '(c', 'd)' ]
\end{DoxyCode}


\subsubsection*{options.\+sep}

{\bfseries Type}\+: {\ttfamily string}

{\bfseries Default}\+: {\ttfamily .}

The separator/character to split on.

{\bfseries Example}


\begin{DoxyCode}
split('a.b,c', \{sep: ','\});
//=> ['a.b', 'c']

// you can also pass the separator as string as the last argument
split('a.b,c', ',');
//=> ['a.b', 'c']
\end{DoxyCode}


\subsubsection*{options.\+keep\+Escaping}

{\bfseries Type}\+: {\ttfamily boolean}

{\bfseries Default}\+: {\ttfamily undefined}

Keep backslashes in the result.

{\bfseries Example}


\begin{DoxyCode}
split('a.b\(\backslash\)\(\backslash\).c');
//=> ['a', 'b.c']

split('a.b.\(\backslash\)\(\backslash\)c', \{keepEscaping: true\});
//=> ['a', 'b\(\backslash\).c']
\end{DoxyCode}


\subsubsection*{options.\+keep\+Quotes}

{\bfseries Type}\+: {\ttfamily boolean}

{\bfseries Default}\+: {\ttfamily undefined}

Keep single-\/ or double-\/quotes in the result.

{\bfseries Example}


\begin{DoxyCode}
split('a."b.c.d".e');
//=> ['a', 'b.c.d', 'e']

split('a."b.c.d".e', \{keepQuotes: true\});
//=> ['a', '"b.c.d"', 'e']

split('a.\(\backslash\)'b.c.d\(\backslash\)'.e', \{keepQuotes: true\});
//=> ['a', '\(\backslash\)'b.c.d\(\backslash\)'', 'e']
\end{DoxyCode}


\subsubsection*{options.\+keep\+Double\+Quotes}

{\bfseries Type}\+: {\ttfamily boolean}

{\bfseries Default}\+: {\ttfamily undefined}

Keep double-\/quotes in the result.

{\bfseries Example}


\begin{DoxyCode}
split('a."b.c.d".e');
//=> ['a', 'b.c.d', 'e']

split('a."b.c.d".e', \{keepDoubleQuotes: true\});
//=> ['a', '"b.c.d"', 'e']
\end{DoxyCode}


\subsubsection*{options.\+keep\+Single\+Quotes}

{\bfseries Type}\+: {\ttfamily boolean}

{\bfseries Default}\+: {\ttfamily undefined}

Keep single-\/quotes in the result.

{\bfseries Example}


\begin{DoxyCode}
split('a.\(\backslash\)'b.c.d\(\backslash\)'.e');
//=> ['a', 'b.c.d', 'e']

split('a.\(\backslash\)'b.c.d\(\backslash\)'.e', \{keepSingleQuotes: true\});
//=> ['a', '\(\backslash\)'b.c.d\(\backslash\)'', 'e']
\end{DoxyCode}


\subsection*{Customizer}

{\bfseries Type}\+: {\ttfamily function}

{\bfseries Default}\+: {\ttfamily undefined}

Pass a function as the last argument to customize how tokens are added to the array.

{\bfseries Example}


\begin{DoxyCode}
var arr = split('a.b', function(tok) \{
  if (tok.arr[tok.arr.length - 1] === 'a') \{
    tok.split = false;
  \}
\});
console.log(arr);
//=> ['a.b']
\end{DoxyCode}


{\bfseries Properties}

The {\ttfamily tok} object has the following properties\+:


\begin{DoxyItemize}
\item {\ttfamily tok.\+val} (string) The current value about to be pushed onto the result array
\item {\ttfamily tok.\+idx} (number) the current index in the string
\item {\ttfamily tok.\+str} (string) the entire string
\item {\ttfamily tok.\+arr} (array) the result array
\end{DoxyItemize}

\subsection*{Release history}

\subsubsection*{v3.\+0.\+0 -\/ 2017-\/06-\/17}

{\bfseries Added}


\begin{DoxyItemize}
\item adds support for brackets
\end{DoxyItemize}

\subsection*{About}

$<$details$>$ 

{\bfseries Contributing}

Pull requests and stars are always welcome. For bugs and feature requests, \href{../../issues/new}{\tt please create an issue}.

$<$/details$>$

$<$details$>$ 

{\bfseries Running Tests}

Running and reviewing unit tests is a great way to get familiarized with a library and its A\+PI. You can install dependencies and run tests with the following command\+:


\begin{DoxyCode}
$ npm install && npm test
\end{DoxyCode}


$<$/details$>$

$<$details$>$ 

{\bfseries Building docs}

\+\_\+(This project\textquotesingle{}s readme.\+md is generated by \href{https://github.com/verbose/verb-generate-readme}{\tt verb}, please don\textquotesingle{}t edit the readme directly. Any changes to the readme must be made in the .verb.\+md \char`\"{}.\+verb.\+md\char`\"{} readme template.)\+\_\+

To generate the readme, run the following command\+:


\begin{DoxyCode}
$ npm install -g verbose/verb#dev verb-generate-readme && verb
\end{DoxyCode}


$<$/details$>$

\subsubsection*{Related projects}

You might also be interested in these projects\+:


\begin{DoxyItemize}
\item \href{https://www.npmjs.com/package/deromanize}{\tt deromanize}\+: Convert roman numerals to arabic numbers (useful for books, outlines, documentation, slide decks, etc) $\vert$ \mbox{[}homepage\mbox{]}(\href{https://github.com/jonschlinkert/deromanize}{\tt https\+://github.\+com/jonschlinkert/deromanize} \char`\"{}\+Convert roman numerals to arabic numbers (useful for books, outlines, documentation, slide decks, etc)\char`\"{})
\item \href{https://www.npmjs.com/package/randomatic}{\tt randomatic}\+: Generate randomized strings of a specified length using simple character sequences. The original generate-\/password. $\vert$ \href{https://github.com/jonschlinkert/randomatic}{\tt homepage}
\item \href{https://www.npmjs.com/package/repeat-string}{\tt repeat-\/string}\+: Repeat the given string n times. Fastest implementation for repeating a string. $\vert$ \href{https://github.com/jonschlinkert/repeat-string}{\tt homepage}
\item \href{https://www.npmjs.com/package/romanize}{\tt romanize}\+: Convert numbers to roman numerals (useful for books, outlines, documentation, slide decks, etc) $\vert$ \mbox{[}homepage\mbox{]}(\href{https://github.com/jonschlinkert/romanize}{\tt https\+://github.\+com/jonschlinkert/romanize} \char`\"{}\+Convert numbers to roman numerals (useful for books, outlines, documentation, slide decks, etc)\char`\"{})
\end{DoxyItemize}

\subsubsection*{Contributors}

\tabulinesep=1mm
\begin{longtabu} spread 0pt [c]{*{2}{|X[-1]}|}
\hline
\rowcolor{\tableheadbgcolor}\multicolumn{2}{|p{(\linewidth-\tabcolsep*2-\arrayrulewidth*1)*2/2}|}{\cellcolor{\tableheadbgcolor}\textbf{ $\ast$$\ast$\+Commits$\ast$   }}\\\cline{1-2}
\endfirsthead
\hline
\endfoot
\hline
\rowcolor{\tableheadbgcolor}\multicolumn{2}{|p{(\linewidth-\tabcolsep*2-\arrayrulewidth*1)*2/2}|}{\cellcolor{\tableheadbgcolor}\textbf{ $\ast$$\ast$\+Commits$\ast$   }}\\\cline{1-2}
\endhead
28  &\href{https://github.com/jonschlinkert}{\tt jonschlinkert}   \\\cline{1-2}
9  &\href{https://github.com/doowb}{\tt doowb}   \\\cline{1-2}
\end{longtabu}


\subsubsection*{Author}

{\bfseries Jon Schlinkert}


\begin{DoxyItemize}
\item \href{https://github.com/jonschlinkert}{\tt github/jonschlinkert}
\item \href{https://twitter.com/jonschlinkert}{\tt twitter/jonschlinkert}
\end{DoxyItemize}

\subsubsection*{License}

Copyright © 2017, \href{https://github.com/jonschlinkert}{\tt Jon Schlinkert}. Released under the \mbox{[}M\+IT License\mbox{]}(L\+I\+C\+E\+N\+SE).





{\itshape This file was generated by \href{https://github.com/verbose/verb-generate-readme}{\tt verb-\/generate-\/readme}, v0.\+6.\+0, on November 19, 2017.} 