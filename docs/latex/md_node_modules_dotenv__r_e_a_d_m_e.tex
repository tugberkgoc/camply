

Dotenv is a zero-\/dependency module that loads environment variables from a {\ttfamily .env} file into \href{https://nodejs.org/docs/latest/api/process.html#process_process_env}{\tt {\ttfamily process.\+env}}. Storing configuration in the environment separate from code is based on \href{http://12factor.net/config}{\tt The Twelve-\/\+Factor App} methodology.

\href{https://travis-ci.org/motdotla/dotenv}{\tt } \href{https://ci.appveyor.com/project/motdotla/dotenv/branch/master}{\tt } \href{https://www.npmjs.com/package/dotenv}{\tt } \href{https://github.com/feross/standard}{\tt } \href{https://coveralls.io/github/motdotla/dotenv?branch=coverall-intergration}{\tt }

\subsection*{Install}


\begin{DoxyCode}
# with npm
npm install dotenv

# or with Yarn
yarn add dotenv
\end{DoxyCode}


\subsection*{Usage}

As early as possible in your application, require and configure dotenv.


\begin{DoxyCode}
require('dotenv').config()
\end{DoxyCode}


Create a {\ttfamily .env} file in the root directory of your project. Add environment-\/specific variables on new lines in the form of {\ttfamily N\+A\+ME=V\+A\+L\+UE}. For example\+:


\begin{DoxyCode}
DB\_HOST=localhost
DB\_USER=root
DB\_PASS=s1mpl3
\end{DoxyCode}


That\textquotesingle{}s it.

{\ttfamily process.\+env} now has the keys and values you defined in your {\ttfamily .env} file.


\begin{DoxyCode}
const db = require('db')
db.connect(\{
  host: process.env.DB\_HOST,
  username: process.env.DB\_USER,
  password: process.env.DB\_PASS
\})
\end{DoxyCode}


\subsubsection*{Preload}

You can use the {\ttfamily -\/-\/require} ({\ttfamily -\/r}) \href{https://nodejs.org/api/cli.html#cli_r_require_module}{\tt command line option} to preload dotenv. By doing this, you do not need to require and load dotenv in your application code. This is the preferred approach when using {\ttfamily import} instead of {\ttfamily require}.


\begin{DoxyCode}
$ node -r dotenv/config your\_script.js
\end{DoxyCode}


The configuration options below are supported as command line arguments in the format {\ttfamily dotenv\+\_\+config\+\_\+$<$option$>$=value}


\begin{DoxyCode}
$ node -r dotenv/config your\_script.js dotenv\_config\_path=/custom/path/to/your/env/vars
\end{DoxyCode}


Additionally, you can use environment variables to set configuration options. Command line arguments will precede these.


\begin{DoxyCode}
$ DOTENV\_CONFIG\_<OPTION>=value node -r dotenv/config your\_script.js
\end{DoxyCode}



\begin{DoxyCode}
$ DOTENV\_CONFIG\_ENCODING=base64 node -r dotenv/config your\_script.js
       dotenv\_config\_path=/custom/path/to/.env
\end{DoxyCode}


\subsection*{Config}

{\itshape Alias\+: {\ttfamily load}}

{\ttfamily config} will read your .env file, parse the contents, assign it to \href{https://nodejs.org/docs/latest/api/process.html#process_process_env}{\tt {\ttfamily process.\+env}}, and return an Object with a {\ttfamily parsed} key containing the loaded content or an {\ttfamily error} key if it failed.


\begin{DoxyCode}
const result = dotenv.config()

if (result.error) \{
  throw result.error
\}

console.log(result.parsed)
\end{DoxyCode}


You can additionally, pass options to {\ttfamily config}.

\subsubsection*{Options}

\paragraph*{Path}

Default\+: `path.\+resolve(process.\+cwd(), '.env\textquotesingle{})\`{}

You may specify a custom path if your file containing environment variables is located elsewhere.


\begin{DoxyCode}
require('dotenv').config(\{ path: '/full/custom/path/to/your/env/vars' \})
\end{DoxyCode}


\paragraph*{Encoding}

Default\+: {\ttfamily utf8}

You may specify the encoding of your file containing environment variables.


\begin{DoxyCode}
require('dotenv').config(\{ encoding: 'base64' \})
\end{DoxyCode}


\paragraph*{Debug}

Default\+: {\ttfamily false}

You may turn on logging to help debug why certain keys or values are not being set as you expect.


\begin{DoxyCode}
require('dotenv').config(\{ debug: process.env.DEBUG \})
\end{DoxyCode}


\subsection*{Parse}

The engine which parses the contents of your file containing environment variables is available to use. It accepts a String or Buffer and will return an Object with the parsed keys and values.


\begin{DoxyCode}
const dotenv = require('dotenv')
const buf = Buffer.from('BASIC=basic')
const config = dotenv.parse(buf) // will return an object
console.log(typeof config, config) // object \{ BASIC : 'basic' \}
\end{DoxyCode}


\subsubsection*{Options}

\paragraph*{Debug}

Default\+: {\ttfamily false}

You may turn on logging to help debug why certain keys or values are not being set as you expect.


\begin{DoxyCode}
const dotenv = require('dotenv')
const buf = Buffer.from('hello world')
const opt = \{ debug: true \}
const config = dotenv.parse(buf, opt)
// expect a debug message because the buffer is not in KEY=VAL form
\end{DoxyCode}


\subsubsection*{Rules}

The parsing engine currently supports the following rules\+:


\begin{DoxyItemize}
\item {\ttfamily B\+A\+S\+IC=basic} becomes `\{B\+A\+S\+IC\+: \textquotesingle{}basic'\}\`{}
\item empty lines are skipped
\item lines beginning with {\ttfamily \#} are treated as comments
\item empty values become empty strings ({\ttfamily E\+M\+P\+TY=} becomes `\{E\+M\+P\+TY\+: '\textquotesingle{}\}{\ttfamily )}
\item {\ttfamily single and double quoted values are escaped (}S\+I\+N\+G\+L\+E\+\_\+\+Q\+U\+O\+TE=\textquotesingle{}quoted\textquotesingle{}{\ttfamily becomes}\{S\+I\+N\+G\+L\+E\+\_\+\+Q\+U\+O\+TE\+: \char`\"{}quoted\char`\"{}\}{\ttfamily )}
\item {\ttfamily new lines are expanded if in double quotes (}M\+U\+L\+T\+I\+L\+I\+NE=\char`\"{}new\textbackslash{}nline\char`\"{}\`{} becomes
\end{DoxyItemize}


\begin{DoxyCode}
\{MULTILINE: 'new
line'\}
\end{DoxyCode}



\begin{DoxyItemize}
\item inner quotes are maintained (think J\+S\+ON) ({\ttfamily J\+S\+ON=\{\char`\"{}foo\char`\"{}\+: \char`\"{}bar\char`\"{}\}} becomes {\ttfamily \{J\+S\+ON\+:\char`\"{}\{\textbackslash{}\char`\"{}foo"\+: "bar"\}\char`\"{}$<$/tt$>$)
-\/ whitespace is removed from both ends of the value (see more on $<$a href=\char`\"{}\href{https://developer.mozilla.org/en-US/docs/Web/JavaScript/Reference/Global_Objects/String/Trim}{\tt https\+://developer.\+mozilla.\+org/en-\/\+U\+S/docs/\+Web/\+Java\+Script/\+Reference/\+Global\+\_\+\+Objects/\+String/\+Trim}\char`\"{}$>$$<$tt$>$trim$<$/tt$>$$<$/a$>$) ($<$tt$>$\+F\+O\+O=\char`\"{} some value "} becomes `\{F\+OO\+: \textquotesingle{}some value'\}\`{})
\end{DoxyItemize}

\subsection*{F\+AQ}

\subsubsection*{Should I commit my {\ttfamily .env} file?}

No. We {\bfseries strongly} recommend against committing your {\ttfamily .env} file to version control. It should only include environment-\/specific values such as database passwords or A\+PI keys. Your production database should have a different password than your development database.

\subsubsection*{Should I have multiple {\ttfamily .env} files?}

No. We {\bfseries strongly} recommend against having a \char`\"{}main\char`\"{} {\ttfamily .env} file and an \char`\"{}environment\char`\"{} {\ttfamily .env} file like {\ttfamily .env.\+test}. Your config should vary between deploys, and you should not be sharing values between environments.

\begin{quote}
In a twelve-\/factor app, env vars are granular controls, each fully orthogonal to other env vars. They are never grouped together as “environments”, but instead are independently managed for each deploy. This is a model that scales up smoothly as the app naturally expands into more deploys over its lifetime.

– \href{http://12factor.net/config}{\tt The Twelve-\/\+Factor App} \end{quote}


\subsubsection*{What happens to environment variables that were already set?}

We will never modify any environment variables that have already been set. In particular, if there is a variable in your {\ttfamily .env} file which collides with one that already exists in your environment, then that variable will be skipped. This behavior allows you to override all {\ttfamily .env} configurations with a machine-\/specific environment, although it is not recommended.

If you want to override {\ttfamily process.\+env} you can do something like this\+:


\begin{DoxyCode}
const fs = require('fs')
const dotenv = require('dotenv')
const envConfig = dotenv.parse(fs.readFileSync('.env.override'))
for (let k in envConfig) \{
  process.env[k] = envConfig[k]
\}
\end{DoxyCode}


\subsubsection*{Can I customize/write plugins for dotenv?}

For {\ttfamily dotenv@2.\+x.\+x}\+: Yes. {\ttfamily dotenv.\+config()} now returns an object representing the parsed {\ttfamily .env} file. This gives you everything you need to continue setting values on {\ttfamily process.\+env}. For example\+:


\begin{DoxyCode}
const dotenv = require('dotenv')
const variableExpansion = require('dotenv-expand')
const myEnv = dotenv.config()
variableExpansion(myEnv)
\end{DoxyCode}


\subsubsection*{What about variable expansion?}

Try \href{https://github.com/motdotla/dotenv-expand}{\tt dotenv-\/expand}

\subsubsection*{How do I use dotenv with {\ttfamily import}?}

E\+S2015 and beyond offers modules that allow you to {\ttfamily export} any top-\/level {\ttfamily function}, {\ttfamily class}, {\ttfamily var}, {\ttfamily let}, or {\ttfamily const}.

\begin{quote}
When you run a module containing an {\ttfamily import} declaration, the modules it imports are loaded first, then each module body is executed in a depth-\/first traversal of the dependency graph, avoiding cycles by skipping anything already executed.

– \href{https://hacks.mozilla.org/2015/08/es6-in-depth-modules/}{\tt E\+S6 In Depth\+: Modules} \end{quote}


You must run {\ttfamily dotenv.\+config()} before referencing any environment variables. Here\textquotesingle{}s an example of problematic code\+:

{\ttfamily error\+Reporter.\+js}\+:


\begin{DoxyCode}
import \{ Client \} from 'best-error-reporting-service'

export const client = new Client(process.env.BEST\_API\_KEY)
\end{DoxyCode}


{\ttfamily index.\+js}\+:


\begin{DoxyCode}
import dotenv from 'dotenv'
import errorReporter from './errorReporter'

dotenv.config()
errorReporter.client.report(new Error('faq example'))
\end{DoxyCode}


{\ttfamily client} will not be configured correctly because it was constructed before {\ttfamily dotenv.\+config()} was executed. There are (at least) 3 ways to make this work.


\begin{DoxyEnumerate}
\item Preload dotenv\+: {\ttfamily node -\/-\/require dotenv/config index.\+js} ({\itshape Note\+: you do not need to {\ttfamily import} dotenv with this approach})
\item Import {\ttfamily dotenv/config} instead of {\ttfamily dotenv} ({\itshape Note\+: you do not need to call {\ttfamily dotenv.\+config()} and must pass options via the command line with this approach})
\item Create a separate file that will execute {\ttfamily config} first as outlined in \href{https://github.com/motdotla/dotenv/issues/133#issuecomment-255298822}{\tt this comment on \#133}
\end{DoxyEnumerate}

\subsection*{Contributing Guide}

See C\+O\+N\+T\+R\+I\+B\+U\+T\+I\+NG.md

\subsection*{Change Log}

See C\+H\+A\+N\+G\+E\+L\+OG.md

\subsection*{License}

See \mbox{[}L\+I\+C\+E\+N\+SE\mbox{]}(L\+I\+C\+E\+N\+SE)

\subsection*{Who\textquotesingle{}s using dotenv}

Here\textquotesingle{}s just a few of many repositories using dotenv\+:


\begin{DoxyItemize}
\item \href{https://github.com/jaws-framework/jaws-core-js}{\tt jaws}
\item \href{https://github.com/motdotla/node-lambda}{\tt node-\/lambda}
\item \href{https://www.npmjs.com/package/resume-cli}{\tt resume-\/cli}
\item \href{https://www.npmjs.com/package/phant}{\tt phant}
\item \href{https://github.com/adafruit/adafruit-io-node}{\tt adafruit-\/io-\/node}
\item \href{https://www.npmjs.com/package/mockbin}{\tt mockbin}
\item \href{https://www.npmjs.com/browse/depended/dotenv}{\tt and many more...}
\end{DoxyItemize}

\subsection*{Go well with dotenv}

Here\textquotesingle{}s some projects that expand on dotenv. Check them out.


\begin{DoxyItemize}
\item \href{https://github.com/bjoshuanoah/require-environment-variables}{\tt require-\/environment-\/variables}
\item \href{https://github.com/rolodato/dotenv-safe}{\tt dotenv-\/safe}
\item \href{https://github.com/af/envalid}{\tt envalid}
\item \href{https://github.com/RodrigoEspinosa/lookenv}{\tt lookenv}
\item \href{https://www.npmjs.com/package/run.env}{\tt run.\+env}
\item \href{https://github.com/mrsteele/dotenv-webpack}{\tt dotenv-\/webpack}
\item \href{https://github.com/benrei/env-path}{\tt env-\/path} 
\end{DoxyItemize}