A robust Punycode converter that fully complies to \href{https://tools.ietf.org/html/rfc3492}{\tt R\+FC 3492} and \href{https://tools.ietf.org/html/rfc5891}{\tt R\+FC 5891}, and works on nearly all Java\+Script platforms.

This Java\+Script library is the result of comparing, optimizing and documenting different open-\/source implementations of the Punycode algorithm\+:


\begin{DoxyItemize}
\item \href{https://tools.ietf.org/html/rfc3492#appendix-C}{\tt The C example code from R\+FC 3492}
\item \href{http://opensource.apple.com/source/ICU/ICU-400.42/icuSources/common/punycode.c}{\tt {\ttfamily punycode.\+c} by {\itshape Markus W. Scherer} (I\+BM)}
\item \href{https://github.com/bnoordhuis/punycode/blob/master/punycode.c}{\tt {\ttfamily punycode.\+c} by {\itshape Ben Noordhuis}}
\item \href{http://stackoverflow.com/questions/183485/can-anyone-recommend-a-good-free-javascript-for-punycode-to-unicode-conversion/301287#301287}{\tt Java\+Script implementation by {\itshape some}}
\item \href{https://github.com/joyent/node/blob/426298c8c1c0d5b5224ac3658c41e7c2a3fe9377/lib/punycode.js}{\tt {\ttfamily punycode.\+js} by {\itshape Ben Noordhuis}} (note\+: \href{https://github.com/joyent/node/issues/2072}{\tt not fully compliant})
\end{DoxyItemize}

This project is \href{https://github.com/joyent/node/blob/master/lib/punycode.js}{\tt bundled} with \href{https://github.com/joyent/node/compare/975f1930b1...61e796decc}{\tt Node.\+js v0.\+6.\+2+} and \href{https://github.com/iojs/io.js/blob/v1.x/lib/punycode.js}{\tt io.\+js v1.\+0.\+0+}.

\subsection*{Installation}

Via \href{https://www.npmjs.com/}{\tt npm} (only required for Node.\+js releases older than v0.\+6.\+2)\+:


\begin{DoxyCode}
npm install punycode
\end{DoxyCode}


Via \href{http://bower.io/}{\tt Bower}\+:


\begin{DoxyCode}
bower install punycode
\end{DoxyCode}


Via \href{https://github.com/component/component}{\tt Component}\+:


\begin{DoxyCode}
component install bestiejs/punycode.js
\end{DoxyCode}


In a browser\+:


\begin{DoxyCode}
<script src="punycode.js"></script>
\end{DoxyCode}


In \href{https://nodejs.org/}{\tt Node.\+js}, \href{https://iojs.org/}{\tt io.\+js}, \href{http://narwhaljs.org/}{\tt Narwhal}, and \href{http://ringojs.org/}{\tt Ringo\+JS}\+:


\begin{DoxyCode}
var punycode = require('punycode');
\end{DoxyCode}


In \href{http://www.mozilla.org/rhino/}{\tt Rhino}\+:


\begin{DoxyCode}
load('punycode.js');
\end{DoxyCode}


Using an A\+MD loader like \href{http://requirejs.org/}{\tt Require\+JS}\+:


\begin{DoxyCode}
require(
  \{
    'paths': \{
      'punycode': 'path/to/punycode'
    \}
  \},
  ['punycode'],
  function(punycode) \{
    console.log(punycode);
  \}
);
\end{DoxyCode}


\subsection*{A\+PI}

\subsubsection*{{\ttfamily punycode.\+decode(string)}}

Converts a Punycode string of A\+S\+C\+II symbols to a string of Unicode symbols.


\begin{DoxyCode}
// decode domain name parts
punycode.decode('maana-pta'); // 'mañana'
punycode.decode('--dqo34k'); // '☃-⌘'
\end{DoxyCode}


\subsubsection*{{\ttfamily punycode.\+encode(string)}}

Converts a string of Unicode symbols to a Punycode string of A\+S\+C\+II symbols.


\begin{DoxyCode}
// encode domain name parts
punycode.encode('mañana'); // 'maana-pta'
punycode.encode('☃-⌘'); // '--dqo34k'
\end{DoxyCode}


\subsubsection*{{\ttfamily punycode.\+to\+Unicode(input)}}

Converts a Punycode string representing a domain name or an email address to Unicode. Only the Punycoded parts of the input will be converted, i.\+e. it doesn’t matter if you call it on a string that has already been converted to Unicode.


\begin{DoxyCode}
// decode domain names
punycode.toUnicode('xn--maana-pta.com');
// → 'mañana.com'
punycode.toUnicode('xn----dqo34k.com');
// → '☃-⌘.com'

// decode email addresses
punycode.toUnicode('джумла@xn--p-8sbkgc5ag7bhce.xn--ba-lmcq');
// → 'джумла@джpумлатест.bрфa'
\end{DoxyCode}


\subsubsection*{{\ttfamily punycode.\+to\+A\+S\+C\+I\+I(input)}}

Converts a lowercased Unicode string representing a domain name or an email address to Punycode. Only the non-\/\+A\+S\+C\+II parts of the input will be converted, i.\+e. it doesn’t matter if you call it with a domain that’s already in A\+S\+C\+II.


\begin{DoxyCode}
// encode domain names
punycode.toASCII('mañana.com');
// → 'xn--maana-pta.com'
punycode.toASCII('☃-⌘.com');
// → 'xn----dqo34k.com'

// encode email addresses
punycode.toASCII('джумла@джpумлатест.bрфa');
// → 'джумла@xn--p-8sbkgc5ag7bhce.xn--ba-lmcq'
\end{DoxyCode}


\subsubsection*{{\ttfamily punycode.\+ucs2}}

\paragraph*{{\ttfamily punycode.\+ucs2.\+decode(string)}}

Creates an array containing the numeric code point values of each Unicode symbol in the string. While \href{https://mathiasbynens.be/notes/javascript-encoding}{\tt Java\+Script uses U\+C\+S-\/2 internally}, this function will convert a pair of surrogate halves (each of which U\+C\+S-\/2 exposes as separate characters) into a single code point, matching U\+T\+F-\/16.


\begin{DoxyCode}
punycode.ucs2.decode('abc');
// → [0x61, 0x62, 0x63]
// surrogate pair for U+1D306 TETRAGRAM FOR CENTRE:
punycode.ucs2.decode('\(\backslash\)uD834\(\backslash\)uDF06');
// → [0x1D306]
\end{DoxyCode}


\paragraph*{{\ttfamily punycode.\+ucs2.\+encode(code\+Points)}}

Creates a string based on an array of numeric code point values.


\begin{DoxyCode}
punycode.ucs2.encode([0x61, 0x62, 0x63]);
// → 'abc'
punycode.ucs2.encode([0x1D306]);
// → '\(\backslash\)uD834\(\backslash\)uDF06'
\end{DoxyCode}


\subsubsection*{{\ttfamily punycode.\+version}}

A string representing the current Punycode.\+js version number.

\subsection*{Unit tests \& code coverage}

After cloning this repository, run {\ttfamily npm install -\/-\/dev} to install the dependencies needed for Punycode.\+js development and testing. You may want to install Istanbul {\itshape globally} using {\ttfamily npm install istanbul -\/g}.

Once that’s done, you can run the unit tests in Node using {\ttfamily npm test} or {\ttfamily node tests/tests.\+js}. To run the tests in Rhino, Ringo, Narwhal, Phantom\+JS, and web browsers as well, use {\ttfamily grunt test}.

To generate the code coverage report, use {\ttfamily grunt cover}.

Feel free to fork if you see possible improvements!

\subsection*{Author}

\tabulinesep=1mm
\begin{longtabu} spread 0pt [c]{*{1}{|X[-1]}|}
\hline
\rowcolor{\tableheadbgcolor}\textbf{ \mbox{[}!   }\\\cline{1-1}
\endfirsthead
\hline
\endfoot
\hline
\rowcolor{\tableheadbgcolor}\textbf{ \mbox{[}!   }\\\cline{1-1}
\endhead
\href{https://mathiasbynens.be/}{\tt Mathias Bynens}   \\\cline{1-1}
\end{longtabu}


\subsection*{Contributors}

\tabulinesep=1mm
\begin{longtabu} spread 0pt [c]{*{1}{|X[-1]}|}
\hline
\rowcolor{\tableheadbgcolor}\textbf{ \mbox{[}!   }\\\cline{1-1}
\endfirsthead
\hline
\endfoot
\hline
\rowcolor{\tableheadbgcolor}\textbf{ \mbox{[}!   }\\\cline{1-1}
\endhead
\href{http://allyoucanleet.com/}{\tt John-\/\+David Dalton}   \\\cline{1-1}
\end{longtabu}


\subsection*{License}

Punycode.\+js is available under the \href{https://mths.be/mit}{\tt M\+IT} license. 