\begin{quote}
Update notifications for your C\+LI app \end{quote}




Inform users of your package of updates in a non-\/intrusive way.

\paragraph*{Contents}


\begin{DoxyItemize}
\item \href{#install}{\tt Install}
\item \href{#usage}{\tt Usage}
\item \href{#how}{\tt How}
\item \href{#api}{\tt A\+PI}
\item \href{#about}{\tt About}
\item \href{#users}{\tt Users}
\end{DoxyItemize}

\subsection*{Install}


\begin{DoxyCode}
$ npm install update-notifier
\end{DoxyCode}


\subsection*{Usage}

\subsubsection*{Simple}


\begin{DoxyCode}
const updateNotifier = require('update-notifier');
const pkg = require('./package.json');

updateNotifier(\{pkg\}).notify();
\end{DoxyCode}


\subsubsection*{Comprehensive}


\begin{DoxyCode}
const updateNotifier = require('update-notifier');
const pkg = require('./package.json');

// Checks for available update and returns an instance
const notifier = updateNotifier(\{pkg\});

// Notify using the built-in convenience method
notifier.notify();

// `notifier.update` contains some useful info about the update
console.log(notifier.update);
/*
\{
  latest: '1.0.1',
  current: '1.0.0',
  type: 'patch', // Possible values: latest, major, minor, patch, prerelease, build
  name: 'pageres'
\}
*/
\end{DoxyCode}


\subsubsection*{Options and custom message}


\begin{DoxyCode}
const notifier = updateNotifier(\{
  pkg,
  updateCheckInterval: 1000 * 60 * 60 * 24 * 7 // 1 week
\});

if (notifier.update) \{
  console.log(`Update available: $\{notifier.update.latest\}`);
\}
\end{DoxyCode}


\subsection*{How}

Whenever you initiate the update notifier and it\textquotesingle{}s not within the interval threshold, it will asynchronously check with npm in the background for available updates, then persist the result. The next time the notifier is initiated, the result will be loaded into the {\ttfamily .update} property. This prevents any impact on your package startup performance. The update check is done in a unref\textquotesingle{}ed \href{https://nodejs.org/api/child_process.html#child_process_child_process_spawn_command_args_options}{\tt child process}. This means that if you call {\ttfamily process.\+exit}, the check will still be performed in its own process.

The first time the user runs your app, it will check for an update, and even if an update is available, it will wait the specified {\ttfamily update\+Check\+Interval} before notifying the user. This is done to not be annoying to the user, but might surprise you as an implementer if you\textquotesingle{}re testing whether it works. Check out \href{example.js}{\tt {\ttfamily example.\+js}} to quickly test out {\ttfamily update-\/notifier} and see how you can test that it works in your app.

\subsection*{A\+PI}

\subsubsection*{notifier = update\+Notifier(options)}

Checks if there is an available update. Accepts options defined below. Returns an instance with an {\ttfamily .update} property there is an available update, otherwise {\ttfamily undefined}.

\subsubsection*{options}

\paragraph*{pkg}

Type\+: {\ttfamily Object}

\subparagraph*{name}

{\itshape Required}~\newline
 Type\+: {\ttfamily string}

\subparagraph*{version}

{\itshape Required}~\newline
 Type\+: {\ttfamily string}

\paragraph*{update\+Check\+Interval}

Type\+: {\ttfamily number}~\newline
 Default\+: {\ttfamily 1000 $\ast$ 60 $\ast$ 60 $\ast$ 24} $\ast$(1 day)$\ast$

How often to check for updates.

\paragraph*{callback(error, update)}

Type\+: {\ttfamily Function}

Passing a callback here will make it check for an update directly and report right away. Not recommended as you won\textquotesingle{}t get the benefits explained in \href{#how}{\tt {\ttfamily How}}. {\ttfamily update} is equal to {\ttfamily notifier.\+update}.

\subsubsection*{notifier.\+notify(\mbox{[}options\mbox{]})}

Convenience method to display a notification message. $\ast$(See screenshot)$\ast$

Only notifies if there is an update and the process is \href{https://nodejs.org/api/process.html#process_tty_terminals_and_process_stdout}{\tt T\+TY}.

\paragraph*{options}

Type\+: {\ttfamily Object}

\subparagraph*{defer}

Type\+: {\ttfamily boolean}~\newline
 Default\+: {\ttfamily true}

Defer showing the notification to after the process has exited.

\subparagraph*{message}

Type\+: {\ttfamily string}~\newline
 Default\+: \href{https://github.com/yeoman/update-notifier#update-notifier-}{\tt See above screenshot}

Message that will be shown when an update is available.

\subparagraph*{is\+Global}

Type\+: {\ttfamily boolean}~\newline
 Default\+: {\ttfamily true}

Include the {\ttfamily -\/g} argument in the default message\textquotesingle{}s {\ttfamily npm i} recommendation. You may want to change this if your C\+LI package can be installed as a dependency of another project, and don\textquotesingle{}t want to recommend a global installation. This option is ignored if you supply your own {\ttfamily message} (see above).

\subparagraph*{boxen\+Opts}

Type\+: {\ttfamily Object}~\newline
 Default\+: `\{padding\+: 1, margin\+: 1, align\+: \textquotesingle{}center', border\+Color\+: \textquotesingle{}yellow\textquotesingle{}, border\+Style\+: \textquotesingle{}round\textquotesingle{}\}\`{} $\ast$(See screenshot)$\ast$

Options object that will be passed to \href{https://github.com/sindresorhus/boxen}{\tt {\ttfamily boxen}}.

\subparagraph*{should\+Notify\+In\+Npm\+Script}

Type\+: {\ttfamily boolean}~\newline
 Default\+: {\ttfamily false}

Allows notification to be shown when running as an npm script.

\subsubsection*{User settings}

Users of your module have the ability to opt-\/out of the update notifier by changing the {\ttfamily opt\+Out} property to {\ttfamily true} in {\ttfamily $\sim$/.config/configstore/update-\/notifier-\/\mbox{[}your-\/module-\/name\mbox{]}.json}. The path is available in {\ttfamily notifier.\+config.\+path}.

Users can also opt-\/out by https\+://github.com/sindresorhus/guides/blob/master/set-\/environment-\/variables.\+md \char`\"{}setting the environment variable\char`\"{} {\ttfamily N\+O\+\_\+\+U\+P\+D\+A\+T\+E\+\_\+\+N\+O\+T\+I\+F\+I\+ER} with any value or by using the {\ttfamily -\/-\/no-\/update-\/notifier} flag on a per run basis.

The check is also skipped on CI automatically.

\subsection*{About}

The idea for this module came from the desire to apply the browser update strategy to C\+LI tools, where everyone is always on the latest version. We first tried automatic updating, which we discovered wasn\textquotesingle{}t popular. This is the second iteration of that idea, but limited to just update notifications.

\subsection*{Users}

There are a bunch projects using it\+:


\begin{DoxyItemize}
\item \href{https://github.com/npm/npm}{\tt npm} -\/ Package manager for Java\+Script
\item \href{http://yeoman.io}{\tt Yeoman} -\/ Modern workflows for modern webapps
\item \href{https://ava.li}{\tt A\+VA} -\/ Simple concurrent test runner
\item \href{https://github.com/xojs/xo}{\tt XO} -\/ Java\+Script happiness style linter
\item \href{https://github.com/sindresorhus/pageres}{\tt Pageres} -\/ Capture website screenshots
\item \href{http://nodegh.io}{\tt Node GH} -\/ Git\+Hub command line tool
\end{DoxyItemize}

\href{https://www.npmjs.org/browse/depended/update-notifier}{\tt And 1600+ more…}

\subsection*{License}

B\+S\+D-\/2-\/\+Clause © Google 