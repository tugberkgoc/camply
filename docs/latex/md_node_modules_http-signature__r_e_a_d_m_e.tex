node-\/http-\/signature is a node.\+js library that has client and server components for Joyent\textquotesingle{}s H\+T\+TP Signature Scheme.

\subsection*{Usage}

Note the example below signs a request with the same key/cert used to start an H\+T\+TP server. This is almost certainly not what you actually want, but is just used to illustrate the A\+PI calls; you will need to provide your own key management in addition to this library.

\subsubsection*{Client}


\begin{DoxyCode}
var fs = require('fs');
var https = require('https');
var httpSignature = require('http-signature');

var key = fs.readFileSync('./key.pem', 'ascii');

var options = \{
  host: 'localhost',
  port: 8443,
  path: '/',
  method: 'GET',
  headers: \{\}
\};

// Adds a 'Date' header in, signs it, and adds the
// 'Authorization' header in.
var req = https.request(options, function(res) \{
  console.log(res.statusCode);
\});


httpSignature.sign(req, \{
  key: key,
  keyId: './cert.pem'
\});

req.end();
\end{DoxyCode}


\subsubsection*{Server}


\begin{DoxyCode}
var fs = require('fs');
var https = require('https');
var httpSignature = require('http-signature');

var options = \{
  key: fs.readFileSync('./key.pem'),
  cert: fs.readFileSync('./cert.pem')
\};

https.createServer(options, function (req, res) \{
  var rc = 200;
  var parsed = httpSignature.parseRequest(req);
  var pub = fs.readFileSync(parsed.keyId, 'ascii');
  if (!httpSignature.verifySignature(parsed, pub))
    rc = 401;

  res.writeHead(rc);
  res.end();
\}).listen(8443);
\end{DoxyCode}


\subsection*{Installation}

\begin{DoxyVerb}npm install http-signature
\end{DoxyVerb}


\subsection*{License}

M\+IT.

\subsection*{Bugs}

See \href{https://github.com/joyent/node-http-signature/issues}{\tt https\+://github.\+com/joyent/node-\/http-\/signature/issues}. 