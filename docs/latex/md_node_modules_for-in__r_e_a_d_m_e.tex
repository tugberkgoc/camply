\begin{quote}
Iterate over the own and inherited enumerable properties of an object, and return an object with properties that evaluate to true from the callback. Exit early by returning {\ttfamily false}. Java\+Script/\+Node.\+js \end{quote}


\subsection*{Install}

Install with \href{https://www.npmjs.com/}{\tt npm}\+:


\begin{DoxyCode}
$ npm install --save for-in
\end{DoxyCode}


\subsection*{Usage}


\begin{DoxyCode}
var forIn = require('for-in');

var obj = \{a: 'foo', b: 'bar', c: 'baz'\};
var values = [];
var keys = [];

forIn(obj, function (value, key, o) \{
  keys.push(key);
  values.push(value);
\});

console.log(keys);
//=> ['a', 'b', 'c'];

console.log(values);
//=> ['foo', 'bar', 'baz'];
\end{DoxyCode}


\subsection*{About}

\subsubsection*{Related projects}


\begin{DoxyItemize}
\item \href{https://www.npmjs.com/package/arr-flatten}{\tt arr-\/flatten}\+: Recursively flatten an array or arrays. This is the fastest implementation of array flatten. $\vert$ \href{https://github.com/jonschlinkert/arr-flatten}{\tt homepage}
\item \href{https://www.npmjs.com/package/collection-map}{\tt collection-\/map}\+: Returns an array of mapped values from an array or object. $\vert$ \href{https://github.com/jonschlinkert/collection-map}{\tt homepage}
\item \href{https://www.npmjs.com/package/for-own}{\tt for-\/own}\+: Iterate over the own enumerable properties of an object, and return an object with properties… \href{https://github.com/jonschlinkert/for-own}{\tt more} $\vert$ \href{https://github.com/jonschlinkert/for-own}{\tt homepage}
\end{DoxyItemize}

\subsubsection*{Contributing}

Pull requests and stars are always welcome. For bugs and feature requests, \href{../../issues/new}{\tt please create an issue}.

\subsubsection*{Contributors}

\tabulinesep=1mm
\begin{longtabu} spread 0pt [c]{*{2}{|X[-1]}|}
\hline
\rowcolor{\tableheadbgcolor}\multicolumn{2}{|p{(\linewidth-\tabcolsep*2-\arrayrulewidth*1)*2/2}|}{\cellcolor{\tableheadbgcolor}\textbf{ $\ast$$\ast$\+Commits$\ast$   }}\\\cline{1-2}
\endfirsthead
\hline
\endfoot
\hline
\rowcolor{\tableheadbgcolor}\multicolumn{2}{|p{(\linewidth-\tabcolsep*2-\arrayrulewidth*1)*2/2}|}{\cellcolor{\tableheadbgcolor}\textbf{ $\ast$$\ast$\+Commits$\ast$   }}\\\cline{1-2}
\endhead
16  &\href{https://github.com/jonschlinkert}{\tt jonschlinkert}   \\\cline{1-2}
2  &\href{https://github.com/paulirish}{\tt paulirish}   \\\cline{1-2}
\end{longtabu}


\subsubsection*{Building docs}

\+\_\+(This project\textquotesingle{}s readme.\+md is generated by \href{https://github.com/verbose/verb-generate-readme}{\tt verb}, please don\textquotesingle{}t edit the readme directly. Any changes to the readme must be made in the .verb.\+md \char`\"{}.\+verb.\+md\char`\"{} readme template.)\+\_\+

To generate the readme, run the following command\+:


\begin{DoxyCode}
$ npm install -g verbose/verb#dev verb-generate-readme && verb
\end{DoxyCode}


\subsubsection*{Running tests}

Running and reviewing unit tests is a great way to get familiarized with a library and its A\+PI. You can install dependencies and run tests with the following command\+:


\begin{DoxyCode}
$ npm install && npm test
\end{DoxyCode}


\subsubsection*{Author}

{\bfseries Jon Schlinkert}


\begin{DoxyItemize}
\item \href{https://github.com/jonschlinkert}{\tt github/jonschlinkert}
\item \href{https://twitter.com/jonschlinkert}{\tt twitter/jonschlinkert}
\end{DoxyItemize}

\subsubsection*{License}

Copyright © 2017, \href{https://github.com/jonschlinkert}{\tt Jon Schlinkert}. Released under the \mbox{[}M\+IT License\mbox{]}(L\+I\+C\+E\+N\+SE).





{\itshape This file was generated by \href{https://github.com/verbose/verb-generate-readme}{\tt verb-\/generate-\/readme}, v0.\+4.\+2, on February 28, 2017.} 