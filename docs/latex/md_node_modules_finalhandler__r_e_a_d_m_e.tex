\href{https://npmjs.org/package/finalhandler}{\tt } \href{https://npmjs.org/package/finalhandler}{\tt } \href{https://nodejs.org/en/download}{\tt } \href{https://travis-ci.org/pillarjs/finalhandler}{\tt } \href{https://coveralls.io/r/pillarjs/finalhandler?branch=master}{\tt }

Node.\+js function to invoke as the final step to respond to H\+T\+TP request.

\subsection*{Installation}

This is a \href{https://nodejs.org/en/}{\tt Node.\+js} module available through the \href{https://www.npmjs.com/}{\tt npm registry}. Installation is done using the \href{https://docs.npmjs.com/getting-started/installing-npm-packages-locally}{\tt {\ttfamily npm install} command}\+:


\begin{DoxyCode}
$ npm install finalhandler
\end{DoxyCode}


\subsection*{A\+PI}


\begin{DoxyCode}
var finalhandler = require('finalhandler')
\end{DoxyCode}


\subsubsection*{finalhandler(req, res, \mbox{[}options\mbox{]})}

Returns function to be invoked as the final step for the given {\ttfamily req} and {\ttfamily res}. This function is to be invoked as {\ttfamily fn(err)}. If {\ttfamily err} is falsy, the handler will write out a 404 response to the {\ttfamily res}. If it is truthy, an error response will be written out to the {\ttfamily res}.

When an error is written, the following information is added to the response\+:


\begin{DoxyItemize}
\item The {\ttfamily res.\+status\+Code} is set from {\ttfamily err.\+status} (or {\ttfamily err.\+status\+Code}). If this value is outside the 4xx or 5xx range, it will be set to 500.
\item The {\ttfamily res.\+status\+Message} is set according to the status code.
\item The body will be the H\+T\+ML of the status code message if {\ttfamily env} is `\textquotesingle{}production'{\ttfamily , otherwise will be}err.\+stack$<$tt$>$.
\item Any headers specified in anerr.\+headers\`{} object.
\end{DoxyItemize}

The final handler will also unpipe anything from {\ttfamily req} when it is invoked.

\paragraph*{options.\+env}

By default, the environment is determined by {\ttfamily N\+O\+D\+E\+\_\+\+E\+NV} variable, but it can be overridden by this option.

\paragraph*{options.\+onerror}

Provide a function to be called with the {\ttfamily err} when it exists. Can be used for writing errors to a central location without excessive function generation. Called as {\ttfamily onerror(err, req, res)}.

\subsection*{Examples}

\subsubsection*{always 404}


\begin{DoxyCode}
var finalhandler = require('finalhandler')
var http = require('http')

var server = http.createServer(function (req, res) \{
  var done = finalhandler(req, res)
  done()
\})

server.listen(3000)
\end{DoxyCode}


\subsubsection*{perform simple action}


\begin{DoxyCode}
var finalhandler = require('finalhandler')
var fs = require('fs')
var http = require('http')

var server = http.createServer(function (req, res) \{
  var done = finalhandler(req, res)

  fs.readFile('index.html', function (err, buf) \{
    if (err) return done(err)
    res.setHeader('Content-Type', 'text/html')
    res.end(buf)
  \})
\})

server.listen(3000)
\end{DoxyCode}


\subsubsection*{use with middleware-\/style functions}


\begin{DoxyCode}
var finalhandler = require('finalhandler')
var http = require('http')
var serveStatic = require('serve-static')

var serve = serveStatic('public')

var server = http.createServer(function (req, res) \{
  var done = finalhandler(req, res)
  serve(req, res, done)
\})

server.listen(3000)
\end{DoxyCode}


\subsubsection*{keep log of all errors}


\begin{DoxyCode}
var finalhandler = require('finalhandler')
var fs = require('fs')
var http = require('http')

var server = http.createServer(function (req, res) \{
  var done = finalhandler(req, res, \{onerror: logerror\})

  fs.readFile('index.html', function (err, buf) \{
    if (err) return done(err)
    res.setHeader('Content-Type', 'text/html')
    res.end(buf)
  \})
\})

server.listen(3000)

function logerror (err) \{
  console.error(err.stack || err.toString())
\}
\end{DoxyCode}


\subsection*{License}

\mbox{[}M\+IT\mbox{]}(L\+I\+C\+E\+N\+SE) 