\begin{quote}
Map {\ttfamily visit} over an array of objects. \end{quote}


\subsection*{Install}

Install with \href{https://www.npmjs.com/}{\tt npm}\+:


\begin{DoxyCode}
$ npm install --save map-visit
\end{DoxyCode}


\subsection*{Usage}


\begin{DoxyCode}
var mapVisit = require('map-visit');
\end{DoxyCode}


\subsection*{What does this do?}

{\bfseries Assign/\+Merge/\+Extend vs. Visit}

Let\textquotesingle{}s say you want to add a {\ttfamily set} method to your application that will\+:


\begin{DoxyItemize}
\item set key-\/value pairs on a {\ttfamily data} object
\item extend objects onto the {\ttfamily data} object
\item extend arrays of objects onto the data object
\end{DoxyItemize}

{\bfseries Example using {\ttfamily extend}}

Here is one way to accomplish this using Lo-\/\+Dash\textquotesingle{}s {\ttfamily extend} (comparable to {\ttfamily Object.\+assign})\+:


\begin{DoxyCode}
var \_ = require('lodash');

var obj = \{
  data: \{\},
  set: function (key, value) \{
    if (Array.isArray(key)) \{
      \_.extend.apply(\_, [obj.data].concat(key));
    \} else if (typeof key === 'object') \{
      \_.extend(obj.data, key);
    \} else \{
      obj.data[key] = value;
    \}
  \}
\};

obj.set('a', 'a');
obj.set([\{b: 'b'\}, \{c: 'c'\}]);
obj.set(\{d: \{e: 'f'\}\});

console.log(obj.data);
//=> \{a: 'a', b: 'b', c: 'c', d: \{ e: 'f' \}\}
\end{DoxyCode}


The above approach works fine for most use cases. However, {\bfseries if you also want to emit an event} each time a property is added to the {\ttfamily data} object, or you want more control over what happens as the object is extended, a better approach would be to use {\ttfamily visit}.

{\bfseries Example using {\ttfamily visit}}

In this approach\+:


\begin{DoxyItemize}
\item when an array is passed to {\ttfamily set}, the {\ttfamily map\+Visit} library calls the {\ttfamily set} method on each object in the array.
\item when an object is passed, {\ttfamily visit} calls {\ttfamily set} on each property in the object.
\end{DoxyItemize}

As a result, the {\ttfamily data} event will be emitted every time a property is added to {\ttfamily data} (events are just an example, you can use this approach to perform any necessary logic every time the method is called).


\begin{DoxyCode}
var mapVisit = require('map-visit');
var visit = require('object-visit');

var obj = \{
  data: \{\},
  set: function (key, value) \{
    if (Array.isArray(key)) \{
      mapVisit(obj, 'set', key);
    \} else if (typeof key === 'object') \{
      visit(obj, 'set', key);
    \} else \{
      // simulate an event-emitter
      console.log('emit', key, value);
      obj.data[key] = value;
    \}
  \}
\};

obj.set('a', 'a');
obj.set([\{b: 'b'\}, \{c: 'c'\}]);
obj.set(\{d: \{e: 'f'\}\});
obj.set(\{g: 'h', i: 'j', k: 'l'\});

console.log(obj.data);
//=> \{a: 'a', b: 'b', c: 'c', d: \{ e: 'f' \}, g: 'h', i: 'j', k: 'l'\}

// events would look something like:
// emit a a
// emit b b
// emit c c
// emit d \{ e: 'f' \}
// emit g h
// emit i j
// emit k l
\end{DoxyCode}


\subsection*{About}

\subsubsection*{Related projects}


\begin{DoxyItemize}
\item \href{https://www.npmjs.com/package/collection-visit}{\tt collection-\/visit}\+: Visit a method over the items in an object, or map visit over the objects… \href{https://github.com/jonschlinkert/collection-visit}{\tt more} $\vert$ \href{https://github.com/jonschlinkert/collection-visit}{\tt homepage}
\item \href{https://www.npmjs.com/package/object-visit}{\tt object-\/visit}\+: Call a specified method on each value in the given object. $\vert$ \href{https://github.com/jonschlinkert/object-visit}{\tt homepage}
\end{DoxyItemize}

\subsubsection*{Contributing}

Pull requests and stars are always welcome. For bugs and feature requests, \href{../../issues/new}{\tt please create an issue}.

\subsubsection*{Contributors}

\tabulinesep=1mm
\begin{longtabu} spread 0pt [c]{*{2}{|X[-1]}|}
\hline
\rowcolor{\tableheadbgcolor}\multicolumn{2}{|p{(\linewidth-\tabcolsep*2-\arrayrulewidth*1)*2/2}|}{\cellcolor{\tableheadbgcolor}\textbf{ $\ast$$\ast$\+Commits$\ast$   }}\\\cline{1-2}
\endfirsthead
\hline
\endfoot
\hline
\rowcolor{\tableheadbgcolor}\multicolumn{2}{|p{(\linewidth-\tabcolsep*2-\arrayrulewidth*1)*2/2}|}{\cellcolor{\tableheadbgcolor}\textbf{ $\ast$$\ast$\+Commits$\ast$   }}\\\cline{1-2}
\endhead
15  &\href{https://github.com/jonschlinkert}{\tt jonschlinkert}   \\\cline{1-2}
7  &\href{https://github.com/doowb}{\tt doowb}   \\\cline{1-2}
\end{longtabu}


\subsubsection*{Building docs}

\+\_\+(This project\textquotesingle{}s readme.\+md is generated by \href{https://github.com/verbose/verb-generate-readme}{\tt verb}, please don\textquotesingle{}t edit the readme directly. Any changes to the readme must be made in the .verb.\+md \char`\"{}.\+verb.\+md\char`\"{} readme template.)\+\_\+

To generate the readme, run the following command\+:


\begin{DoxyCode}
$ npm install -g verbose/verb#dev verb-generate-readme && verb
\end{DoxyCode}


\subsubsection*{Running tests}

Running and reviewing unit tests is a great way to get familiarized with a library and its A\+PI. You can install dependencies and run tests with the following command\+:


\begin{DoxyCode}
$ npm install && npm test
\end{DoxyCode}


\subsubsection*{Author}

{\bfseries Jon Schlinkert}


\begin{DoxyItemize}
\item \href{https://github.com/jonschlinkert}{\tt github/jonschlinkert}
\item \href{https://twitter.com/jonschlinkert}{\tt twitter/jonschlinkert}
\end{DoxyItemize}

\subsubsection*{License}

Copyright © 2017, \href{https://github.com/jonschlinkert}{\tt Jon Schlinkert}. Released under the \mbox{[}M\+IT License\mbox{]}(L\+I\+C\+E\+N\+SE).





{\itshape This file was generated by \href{https://github.com/verbose/verb-generate-readme}{\tt verb-\/generate-\/readme}, v0.\+5.\+0, on April 09, 2017.} 