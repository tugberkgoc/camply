\begin{quote}
\href{http://en.wikipedia.org/wiki/ANSI_escape_code#Colors_and_Styles}{\tt A\+N\+SI escape codes} for styling strings in the terminal \end{quote}


You probably want the higher-\/level \href{https://github.com/chalk/chalk}{\tt chalk} module for styling your strings.



\subsection*{Install}


\begin{DoxyCode}
$ npm install ansi-styles
\end{DoxyCode}


\subsection*{Usage}


\begin{DoxyCode}
const style = require('ansi-styles');

console.log(`$\{style.green.open\}Hello world!$\{style.green.close\}`);


// Color conversion between 16/256/truecolor
// NOTE: If conversion goes to 16 colors or 256 colors, the original color
//       may be degraded to fit that color palette. This means terminals
//       that do not support 16 million colors will best-match the
//       original color.
console.log(style.bgColor.ansi.hsl(120, 80, 72) + 'Hello world!' + style.bgColor.close);
console.log(style.color.ansi256.rgb(199, 20, 250) + 'Hello world!' + style.color.close);
console.log(style.color.ansi16m.hex('#ABCDEF') + 'Hello world!' + style.color.close);
\end{DoxyCode}


\subsection*{A\+PI}

Each style has an {\ttfamily open} and {\ttfamily close} property.

\subsection*{Styles}

\subsubsection*{Modifiers}


\begin{DoxyItemize}
\item {\ttfamily reset}
\item {\ttfamily bold}
\item {\ttfamily dim}
\item {\ttfamily italic} $\ast$(Not widely supported)$\ast$
\item {\ttfamily underline}
\item {\ttfamily inverse}
\item {\ttfamily hidden}
\item {\ttfamily strikethrough} $\ast$(Not widely supported)$\ast$
\end{DoxyItemize}

\subsubsection*{Colors}


\begin{DoxyItemize}
\item {\ttfamily black}
\item {\ttfamily red}
\item {\ttfamily green}
\item {\ttfamily yellow}
\item {\ttfamily blue}
\item {\ttfamily magenta}
\item {\ttfamily cyan}
\item {\ttfamily white}
\item {\ttfamily gray} (\char`\"{}bright black\char`\"{})
\item {\ttfamily red\+Bright}
\item {\ttfamily green\+Bright}
\item {\ttfamily yellow\+Bright}
\item {\ttfamily blue\+Bright}
\item {\ttfamily magenta\+Bright}
\item {\ttfamily cyan\+Bright}
\item {\ttfamily white\+Bright}
\end{DoxyItemize}

\subsubsection*{Background colors}


\begin{DoxyItemize}
\item {\ttfamily bg\+Black}
\item {\ttfamily bg\+Red}
\item {\ttfamily bg\+Green}
\item {\ttfamily bg\+Yellow}
\item {\ttfamily bg\+Blue}
\item {\ttfamily bg\+Magenta}
\item {\ttfamily bg\+Cyan}
\item {\ttfamily bg\+White}
\item {\ttfamily bg\+Black\+Bright}
\item {\ttfamily bg\+Red\+Bright}
\item {\ttfamily bg\+Green\+Bright}
\item {\ttfamily bg\+Yellow\+Bright}
\item {\ttfamily bg\+Blue\+Bright}
\item {\ttfamily bg\+Magenta\+Bright}
\item {\ttfamily bg\+Cyan\+Bright}
\item {\ttfamily bg\+White\+Bright}
\end{DoxyItemize}

\subsection*{Advanced usage}

By default, you get a map of styles, but the styles are also available as groups. They are non-\/enumerable so they don\textquotesingle{}t show up unless you access them explicitly. This makes it easier to expose only a subset in a higher-\/level module.


\begin{DoxyItemize}
\item {\ttfamily style.\+modifier}
\item {\ttfamily style.\+color}
\item {\ttfamily style.\+bg\+Color}
\end{DoxyItemize}

\subparagraph*{Example}


\begin{DoxyCode}
console.log(style.color.green.open);
\end{DoxyCode}


Raw escape codes (i.\+e. without the C\+SI escape prefix {\ttfamily \textbackslash{}u001B\mbox{[}} and render mode postfix {\ttfamily m}) are available under {\ttfamily style.\+codes}, which returns a {\ttfamily Map} with the open codes as keys and close codes as values.

\subparagraph*{Example}


\begin{DoxyCode}
console.log(style.codes.get(36));
//=> 39
\end{DoxyCode}


\subsection*{\href{https://gist.github.com/XVilka/8346728}{\tt 256 / 16 million (True\+Color) support}}

{\ttfamily ansi-\/styles} uses the \href{https://github.com/Qix-/color-convert}{\tt {\ttfamily color-\/convert}} package to allow for converting between various colors and A\+N\+SI escapes, with support for 256 and 16 million colors.

To use these, call the associated conversion function with the intended output, for example\+:


\begin{DoxyCode}
style.color.ansi.rgb(100, 200, 15); // RGB to 16 color ansi foreground code
style.bgColor.ansi.rgb(100, 200, 15); // RGB to 16 color ansi background code

style.color.ansi256.hsl(120, 100, 60); // HSL to 256 color ansi foreground code
style.bgColor.ansi256.hsl(120, 100, 60); // HSL to 256 color ansi foreground code

style.color.ansi16m.hex('#C0FFEE'); // Hex (RGB) to 16 million color foreground code
style.bgColor.ansi16m.hex('#C0FFEE'); // Hex (RGB) to 16 million color background code
\end{DoxyCode}


\subsection*{Related}


\begin{DoxyItemize}
\item \href{https://github.com/sindresorhus/ansi-escapes}{\tt ansi-\/escapes} -\/ A\+N\+SI escape codes for manipulating the terminal
\end{DoxyItemize}

\subsection*{Maintainers}


\begin{DoxyItemize}
\item \href{https://github.com/sindresorhus}{\tt Sindre Sorhus}
\item \href{https://github.com/qix-}{\tt Josh Junon}
\end{DoxyItemize}

\subsection*{License}

M\+IT 