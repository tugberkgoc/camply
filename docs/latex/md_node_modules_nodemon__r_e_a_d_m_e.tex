

\section*{nodemon}

nodemon is a tool that helps develop node.\+js based applications by automatically restarting the node application when file changes in the directory are detected.

nodemon does {\bfseries not} require {\itshape any} additional changes to your code or method of development. nodemon is a replacement wrapper for {\ttfamily node}, to use {\ttfamily nodemon} replace the word {\ttfamily node} on the command line when executing your script.

\href{https://npmjs.org/package/nodemon}{\tt } \href{https://travis-ci.org/remy/nodemon}{\tt } \href{#backers}{\tt } \href{#sponsors}{\tt }

\section*{Installation}

Either through cloning with git or by using \href{http://npmjs.org}{\tt npm} (the recommended way)\+:


\begin{DoxyCode}
npm install -g nodemon
\end{DoxyCode}


And nodemon will be installed globally to your system path.

You can also install nodemon as a development dependency\+:


\begin{DoxyCode}
npm install --save-dev nodemon
\end{DoxyCode}


With a local installation, nodemon will not be available in your system path. Instead, the local installation of nodemon can be run by calling it from within an npm script (such as {\ttfamily npm start}) or using {\ttfamily npx nodemon}.

\section*{Usage}

nodemon wraps your application, so you can pass all the arguments you would normally pass to your app\+:


\begin{DoxyCode}
nodemon [your node app]
\end{DoxyCode}


For C\+LI options, use the {\ttfamily -\/h} (or {\ttfamily -\/-\/help}) argument\+:


\begin{DoxyCode}
nodemon -h
\end{DoxyCode}


Using nodemon is simple, if my application accepted a host and port as the arguments, I would start it as so\+:


\begin{DoxyCode}
nodemon ./server.js localhost 8080
\end{DoxyCode}


Any output from this script is prefixed with {\ttfamily \mbox{[}nodemon\mbox{]}}, otherwise all output from your application, errors included, will be echoed out as expected.

If no script is given, nodemon will test for a {\ttfamily package.\+json} file and if found, will run the file associated with the {\itshape main} property (\href{https://github.com/remy/nodemon/issues/14}{\tt ref}).

You can also pass the {\ttfamily inspect} flag to node through the command line as you would normally\+:


\begin{DoxyCode}
nodemon --inspect ./server.js 80
\end{DoxyCode}


If you have a {\ttfamily package.\+json} file for your app, you can omit the main script entirely and nodemon will read the {\ttfamily package.\+json} for the {\ttfamily main} property and use that value as the app.

nodemon will also search for the {\ttfamily scripts.\+start} property in {\ttfamily package.\+json} (as of nodemon 1.\+1.\+x).

Also check out the https\+://github.com/remy/nodemon/blob/master/faq.\+md \char`\"{}\+F\+A\+Q\char`\"{} or \href{https://github.com/remy/nodemon/issues}{\tt issues} for nodemon.

\subsection*{Automatic re-\/running}

nodemon was originally written to restart hanging processes such as web servers, but now supports apps that cleanly exit. If your script exits cleanly, nodemon will continue to monitor the directory (or directories) and restart the script if there are any changes.

\subsection*{Manual restarting}

Whilst nodemon is running, if you need to manually restart your application, instead of stopping and restart nodemon, you can type {\ttfamily rs} with a carriage return, and nodemon will restart your process.

\subsection*{Config files}

nodemon supports local and global configuration files. These are usually named {\ttfamily nodemon.\+json} and can be located in the current working directory or in your home directory. An alternative local configuration file can be specified with the {\ttfamily -\/-\/config $<$file$>$} option.

The specificity is as follows, so that a command line argument will always override the config file settings\+:


\begin{DoxyItemize}
\item command line arguments
\item local config
\item global config
\end{DoxyItemize}

A config file can take any of the command line arguments as J\+S\+ON key values, for example\+:


\begin{DoxyCode}
\{
  "verbose": true,
  "ignore": ["*.test.js", "fixtures/*"],
  "execMap": \{
    "rb": "ruby",
    "pde": "processing --sketch=\{\{pwd\}\} --run"
  \}
\}
\end{DoxyCode}


The above {\ttfamily nodemon.\+json} file might be my global config so that I have support for ruby files and processing files, and I can run {\ttfamily nodemon demo.\+pde} and nodemon will automatically know how to run the script even though out of the box support for processing scripts.

A further example of options can be seen in https\+://github.com/remy/nodemon/blob/master/doc/sample-\/nodemon.\+md \char`\"{}sample-\/nodemon.\+md\char`\"{}

\subsubsection*{package.\+json}

If you want to keep all your package configurations in one place, nodemon supports using {\ttfamily package.\+json} for configuration. Specify the config in the same format as you would for a config file but under {\ttfamily nodemon\+Config} in the {\ttfamily package.\+json} file, for example, take the following {\ttfamily package.\+json}\+:


\begin{DoxyCode}
\{
  "name": "nodemon",
  "homepage": "http://nodemon.io",
  "...": "... other standard package.json values",
  "nodemonConfig": \{
    "ignore": ["test/*", "docs/*"],
    "delay": "2500"
  \}
\}
\end{DoxyCode}


Note that if you specify a {\ttfamily -\/-\/config} file or provide a local {\ttfamily nodemon.\+json} any {\ttfamily package.\+json} config is ignored.

{\itshape This section needs better documentation, but for now you can also see {\ttfamily nodemon -\/-\/help config} (\href{https://github.com/remy/nodemon/blob/master/doc/cli/config.txt}{\tt also here})}.

\subsection*{Using nodemon as a module}

Please see doc/requireable.md

\subsection*{Using nodemon as child process}

Please see \href{doc/events.md#Using_nodemon_as_child_process}{\tt doc/events.\+md}

\subsection*{Running non-\/node scripts}

nodemon can also be used to execute and monitor other programs. nodemon will read the file extension of the script being run and monitor that extension instead of {\ttfamily .js} if there\textquotesingle{}s no {\ttfamily nodemon.\+json}\+:


\begin{DoxyCode}
nodemon --exec "python -v" ./app.py
\end{DoxyCode}


Now nodemon will run {\ttfamily app.\+py} with python in verbose mode (note that if you\textquotesingle{}re not passing args to the exec program, you don\textquotesingle{}t need the quotes), and look for new or modified files with the {\ttfamily .py} extension.

\subsubsection*{Default executables}

Using the {\ttfamily nodemon.\+json} config file, you can define your own default executables using the {\ttfamily exec\+Map} property. This is particularly useful if you\textquotesingle{}re working with a language that isn\textquotesingle{}t supported by default by nodemon.

To add support for nodemon to know about the {\ttfamily .pl} extension (for Perl), the {\ttfamily nodemon.\+json} file would add\+:


\begin{DoxyCode}
\{
  "execMap": \{
    "pl": "perl"
  \}
\}
\end{DoxyCode}


Now running the following, nodemon will know to use {\ttfamily perl} as the executable\+:


\begin{DoxyCode}
nodemon script.pl
\end{DoxyCode}


It\textquotesingle{}s generally recommended to use the global {\ttfamily nodemon.\+json} to add your own {\ttfamily exec\+Map} options. However, if there\textquotesingle{}s a common default that\textquotesingle{}s missing, this can be merged in to the project so that nodemon supports it by default, by changing \href{https://github.com/remy/nodemon/blob/master/lib/config/defaults.js}{\tt default.\+js} and sending a pull request.

\subsection*{Monitoring multiple directories}

By default nodemon monitors the current working directory. If you want to take control of that option, use the {\ttfamily -\/-\/watch} option to add specific paths\+:


\begin{DoxyCode}
nodemon --watch app --watch libs app/server.js
\end{DoxyCode}


Now nodemon will only restart if there are changes in the {\ttfamily ./app} or {\ttfamily ./libs} directory. By default nodemon will traverse sub-\/directories, so there\textquotesingle{}s no need in explicitly including sub-\/directories.

Don\textquotesingle{}t use unix globbing to pass multiple directories, e.\+g {\ttfamily -\/-\/watch ./lib/$\ast$}, it won\textquotesingle{}t work. You need a {\ttfamily -\/-\/watch} flag per directory watched.

\subsection*{Specifying extension watch list}

By default, nodemon looks for files with the {\ttfamily .js}, {\ttfamily .mjs}, {\ttfamily .coffee}, {\ttfamily .litcoffee}, and {\ttfamily .json} extensions. If you use the {\ttfamily -\/-\/exec} option and monitor {\ttfamily app.\+py} nodemon will monitor files with the extension of {\ttfamily .py}. However, you can specify your own list with the {\ttfamily -\/e} (or {\ttfamily -\/-\/ext}) switch like so\+:


\begin{DoxyCode}
nodemon -e js,jade
\end{DoxyCode}


Now nodemon will restart on any changes to files in the directory (or subdirectories) with the extensions {\ttfamily .js}, {\ttfamily .jade}.

\subsection*{Ignoring files}

By default, nodemon will only restart when a {\ttfamily .js} Java\+Script file changes. In some cases you will want to ignore some specific files, directories or file patterns, to prevent nodemon from prematurely restarting your application.

This can be done via the command line\+:


\begin{DoxyCode}
nodemon --ignore lib/ --ignore tests/
\end{DoxyCode}


Or specific files can be ignored\+:


\begin{DoxyCode}
nodemon --ignore lib/app.js
\end{DoxyCode}


Patterns can also be ignored (but be sure to quote the arguments)\+:


\begin{DoxyCode}
nodemon --ignore 'lib/*.js'
\end{DoxyCode}


Note that by default, nodemon will ignore the {\ttfamily .git}, {\ttfamily node\+\_\+modules}, {\ttfamily bower\+\_\+components}, {\ttfamily .nyc\+\_\+output}, {\ttfamily coverage} and {\ttfamily .sass-\/cache} directories and {\itshape add} your ignored patterns to the list. If you want to indeed watch a directory like {\ttfamily node\+\_\+modules}, you need to \href{https://github.com/remy/nodemon/blob/master/faq.md#overriding-the-underlying-default-ignore-rules}{\tt override the underlying default ignore rules}.

\subsection*{Application isn\textquotesingle{}t restarting}

In some networked environments (such as a container running nodemon reading across a mounted drive), you will need to use the {\ttfamily legacy\+Watch\+: true} which enables Chokidar\textquotesingle{}s polling.

Via the C\+LI, use either {\ttfamily -\/-\/legacy-\/watch} or {\ttfamily -\/L} for short\+:


\begin{DoxyCode}
nodemon -L
\end{DoxyCode}


Though this should be a last resort as it will poll every file it can find.

\subsection*{Delaying restarting}

In some situations, you may want to wait until a number of files have changed. The timeout before checking for new file changes is 1 second. If you\textquotesingle{}re uploading a number of files and it\textquotesingle{}s taking some number of seconds, this could cause your app to restart multiple times unnecessarily.

To add an extra throttle, or delay restarting, use the {\ttfamily -\/-\/delay} command\+:


\begin{DoxyCode}
nodemon --delay 10 server.js
\end{DoxyCode}


For more precision, milliseconds can be specified. Either as a float\+:


\begin{DoxyCode}
nodemon --delay 2.5 server.js
\end{DoxyCode}


Or using the time specifier (ms)\+:


\begin{DoxyCode}
nodemon --delay 2500ms server.js
\end{DoxyCode}


The delay figure is number of seconds (or milliseconds, if specified) to delay before restarting. So nodemon will only restart your app the given number of seconds after the {\itshape last} file change.

If you are setting this value in {\ttfamily nodemon.\+json}, the value will always be interpreted in milliseconds. E.\+g., the following are equivalent\+:


\begin{DoxyCode}
nodemon --delay 2.5

\{
  "delay": "2500"
\}
\end{DoxyCode}


\subsection*{Gracefully reloading down your script}

It is possible to have nodemon send any signal that you specify to your application.


\begin{DoxyCode}
nodemon --signal SIGHUP server.js
\end{DoxyCode}


Your application can handle the signal as follows.


\begin{DoxyCode}
process.once("SIGHUP", function () \{
  reloadSomeConfiguration();
\})
\end{DoxyCode}


Please note that nodemon will send this signal to every process in the process tree.

If you are using {\ttfamily cluster}, then each workers (as well as the master) will receive the signal. If you wish to terminate all workers on receiving a {\ttfamily S\+I\+G\+H\+UP}, a common pattern is to catch the {\ttfamily S\+I\+G\+H\+UP} in the master, and forward {\ttfamily S\+I\+G\+T\+E\+RM} to all workers, while ensuring that all workers ignore {\ttfamily S\+I\+G\+H\+UP}.


\begin{DoxyCode}
if (cluster.isMaster) \{
  process.on("SIGHUP", function () \{
    for (const worker of Object.values(cluster.workers)) \{
      worker.process.kill("SIGTERM");
    \}
  \});
\} else \{
  process.on("SIGHUP", function() \{\})
\}
\end{DoxyCode}


\subsection*{Controlling shutdown of your script}

nodemon sends a kill signal to your application when it sees a file update. If you need to clean up on shutdown inside your script you can capture the kill signal and handle it yourself.

The following example will listen once for the {\ttfamily S\+I\+G\+U\+S\+R2} signal (used by nodemon to restart), run the clean up process and then kill itself for nodemon to continue control\+:


\begin{DoxyCode}
process.once('SIGUSR2', function () \{
  gracefulShutdown(function () \{
    process.kill(process.pid, 'SIGUSR2');
  \});
\});
\end{DoxyCode}


Note that the {\ttfamily process.\+kill} is {\itshape only} called once your shutdown jobs are complete. Hat tip to \href{http://www.benjiegillam.com/2011/08/node-js-clean-restart-and-faster-development-with-nodemon/}{\tt Benjie Gillam} for writing this technique up.

\subsection*{Triggering events when nodemon state changes}

If you want growl like notifications when nodemon restarts or to trigger an action when an event happens, then you can either {\ttfamily require} nodemon or add event actions to your {\ttfamily nodemon.\+json} file.

For example, to trigger a notification on a Mac when nodemon restarts, {\ttfamily nodemon.\+json} looks like this\+:


\begin{DoxyCode}
\{
  "events": \{
    "restart": "osascript -e 'display notification \(\backslash\)"app restarted\(\backslash\)" with title \(\backslash\)"nodemon\(\backslash\)"'"
  \}
\}
\end{DoxyCode}


A full list of available events is listed on the \href{https://github.com/remy/nodemon/wiki/Events#states}{\tt event states wiki}. Note that you can bind to both states and messages.

\subsection*{Pipe output to somewhere else}


\begin{DoxyCode}
nodemon(\{
  script: ...,
  stdout: false // important: this tells nodemon not to output to console
\}).on('readable', function() \{ // the `readable` event indicates that data is ready to pick up
  this.stdout.pipe(fs.createWriteStream('output.txt'));
  this.stderr.pipe(fs.createWriteStream('err.txt'));
\});
\end{DoxyCode}


\subsection*{Using nodemon in your gulp workflow}

Check out the \href{https://github.com/JacksonGariety/gulp-nodemon}{\tt gulp-\/nodemon} plugin to integrate nodemon with the rest of your project\textquotesingle{}s gulp workflow.

\subsection*{Using nodemon in your Grunt workflow}

Check out the \href{https://github.com/ChrisWren/grunt-nodemon}{\tt grunt-\/nodemon} plugin to integrate nodemon with the rest of your project\textquotesingle{}s grunt workflow.

\subsection*{Pronunciation}

\begin{quote}
nodemon, is it pronounced\+: node-\/mon, no-\/demon or node-\/e-\/mon (like pokémon)? \end{quote}


Well...I\textquotesingle{}ve been asked this many times before. I like that I\textquotesingle{}ve been asked this before. There\textquotesingle{}s been bets as to which one it actually is.

The answer is simple, but possibly frustrating. I\textquotesingle{}m not saying (how I pronounce it). It\textquotesingle{}s up to you to call it as you like. All answers are correct \+:)

\subsection*{Design principles}


\begin{DoxyItemize}
\item Less flags is better
\item Works across all platforms
\item Less features
\item Let individuals build on top of nodemon
\item Offer all C\+LI functionality as an A\+PI
\item Contributions must have and pass tests
\end{DoxyItemize}

Nodemon is not perfect, and C\+LI arguments has sprawled beyond where I\textquotesingle{}m completely happy, but perhaps it can be reduced a little one day.

\subsection*{F\+AQ}

See the https\+://github.com/remy/nodemon/blob/master/faq.\+md \char`\"{}\+F\+A\+Q\char`\"{} and please add your own questions if you think they would help others.

\subsection*{Backers}

Thank you to all \href{https://opencollective.com/nodemon#backer}{\tt our backers}! 🙏

\href{https://opencollective.com/nodemon#backers}{\tt }

\subsection*{Sponsors}

Support this project by becoming a sponsor. Your logo will show up here with a link to your website. \href{https://opencollective.com/nodemon#sponsor}{\tt Sponsor this project today ❤️}

\href{https://sparkpo.st/nodemon}{\tt }

\href{https://mixmax.com}{\tt }

\section*{License}

M\+IT \href{http://rem.mit-license.org}{\tt http\+://rem.\+mit-\/license.\+org} 