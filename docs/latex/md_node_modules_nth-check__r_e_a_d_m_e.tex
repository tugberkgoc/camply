A performant nth-\/check parser \& compiler.

\subsubsection*{About}

This module can be used to parse \& compile nth-\/checks, as they are found in C\+SS 3\textquotesingle{}s {\ttfamily nth-\/child()} and {\ttfamily nth-\/last-\/of-\/type()}.

{\ttfamily nth-\/check} focusses on speed, providing optimized functions for different kinds of nth-\/child formulas, while still following the \href{http://www.w3.org/TR/css3-selectors/#nth-child-pseudo}{\tt spec}.

\subsubsection*{A\+PI}


\begin{DoxyCode}
var nthCheck = require("nth-check");
\end{DoxyCode}


\subparagraph*{{\ttfamily nth\+Check(formula)}}

First parses, then compiles the formula.

\subparagraph*{{\ttfamily nth\+Check.\+parse(formula)}}

Parses the expression, throws a {\ttfamily Syntax\+Error} if it fails, otherwise returns an array containing two elements.

{\bfseries Example\+:}


\begin{DoxyCode}
nthCheck.parse("2n+3") //[2, 3]
\end{DoxyCode}


\subparagraph*{{\ttfamily nth\+Check.\+compile(\mbox{[}a, b\mbox{]})}}

Takes an array with two elements (as returned by {\ttfamily .parse}) and returns a highly optimized function.

If the formula doesn\textquotesingle{}t match any elements, it returns \href{https://github.com/fb55/boolbase}{\tt {\ttfamily boolbase}}\textquotesingle{}s {\ttfamily false\+Func}, otherwise, a function accepting an {\itshape index} is returned, which returns whether or not a passed {\itshape index} matches the formula. (Note\+: The spec starts counting at {\ttfamily 1}, the returned function at {\ttfamily 0}).

{\bfseries Example\+:} 
\begin{DoxyCode}
var check = nthCheck.compile([2, 3]);

check(0) //false
check(1) //false
check(2) //true
check(3) //false
check(4) //true
check(5) //false
check(6) //true
\end{DoxyCode}
 

 License\+: B\+SD 