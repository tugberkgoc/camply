\href{http://travis-ci.org/kevinbeaty/any-promise}{\tt }

Let your library support any ES 2015 (E\+S6) compatible {\ttfamily Promise} and leave the choice to application authors. The application can {\itshape optionally} register its preferred {\ttfamily Promise} implementation and it will be exported when requiring {\ttfamily any-\/promise} from library code.

If no preference is registered, defaults to the global {\ttfamily Promise} for newer Node.\+js versions. The browser version defaults to the window {\ttfamily Promise}, so polyfill or register as necessary.

\subsubsection*{Usage with global Promise\+:}

Assuming the global {\ttfamily Promise} is the desired implementation\+:


\begin{DoxyCode}
# Install any libraries depending on any-promise
$ npm install mz
\end{DoxyCode}


The installed libraries will use global Promise by default.


\begin{DoxyCode}
// in library
var Promise = require('any-promise')  // the global Promise

function promiseReturningFunction()\{
    return new Promise(function(resolve, reject)\{...\})
\}
\end{DoxyCode}


\subsubsection*{Usage with registration\+:}

Assuming {\ttfamily bluebird} is the desired Promise implementation\+:


\begin{DoxyCode}
# Install preferred promise library
$ npm install bluebird
# Install any-promise to allow registration
$ npm install any-promise
# Install any libraries you would like to use depending on any-promise
$ npm install mz
\end{DoxyCode}


Register your preference in the application entry point before any other {\ttfamily require} of packages that load {\ttfamily any-\/promise}\+:


\begin{DoxyCode}
// top of application index.js or other entry point
require('any-promise/register/bluebird')

// -or- Equivalent to above, but allows customization of Promise library
require('any-promise/register')('bluebird', \{Promise: require('bluebird')\})
\end{DoxyCode}


Now that the implementation is registered, you can use any package depending on {\ttfamily any-\/promise}\+:


\begin{DoxyCode}
var fsp = require('mz/fs') // mz/fs will use registered bluebird promises
var Promise = require('any-promise')  // the registered bluebird promise 
\end{DoxyCode}


It is safe to call {\ttfamily register} multiple times, but it must always be with the same implementation.

Again, registration is {\itshape optional}. It should only be called by the application user if overriding the global {\ttfamily Promise} implementation is desired.

\subsubsection*{Optional Application Registration}

As an application author, you can {\itshape optionally} register a preferred {\ttfamily Promise} implementation on application startup (before any call to `require(\textquotesingle{}any-\/promise')\`{}\+:

You must register your preference before any call to `require(\textquotesingle{}any-\/promise')\`{} (by you or required packages), and only one implementation can be registered. Typically, this registration would occur at the top of the application entry point.

\paragraph*{Registration shortcuts}

If you are using a known {\ttfamily Promise} implementation, you can register your preference with a shortcut\+:


\begin{DoxyCode}
require('any-promise/register/bluebird')
// -or-
import 'any-promise/register/q';
\end{DoxyCode}


Shortcut registration is the preferred registration method as it works in the browser and Node.\+js. It is also convenient for using with {\ttfamily import} and many test runners, that offer a {\ttfamily -\/-\/require} flag\+:


\begin{DoxyCode}
$ ava --require=any-promise/register/bluebird test.js
\end{DoxyCode}


Current known implementations include {\ttfamily bluebird}, {\ttfamily q}, {\ttfamily when}, {\ttfamily rsvp}, {\ttfamily es6-\/promise}, {\ttfamily promise}, {\ttfamily native-\/promise-\/only}, {\ttfamily pinkie}, {\ttfamily vow} and {\ttfamily lie}. If you are not using a known implementation, you can use another registration method described below.

\paragraph*{Basic Registration}

As an alternative to registration shortcuts, you can call the {\ttfamily register} function with the preferred {\ttfamily Promise} implementation. The benefit of this approach is that a {\ttfamily Promise} library can be required by name without being a known implementation. This approach does N\+OT work in the browser. To use {\ttfamily any-\/promise} in the browser use either registration shortcuts or specify the {\ttfamily Promise} constructor using advanced registration (see below).


\begin{DoxyCode}
require('any-promise/register')('when')
// -or- require('any-promise/register')('any other ES6 compatible library (known or otherwise)')
\end{DoxyCode}


This registration method will try to detect the {\ttfamily Promise} constructor from requiring the specified implementation. If you would like to specify your own constructor, see advanced registration.

\paragraph*{Advanced Registration}

To use the browser version, you should either install a polyfill or explicitly register the {\ttfamily Promise} constructor\+:


\begin{DoxyCode}
require('any-promise/register')('bluebird', \{Promise: require('bluebird')\})
\end{DoxyCode}


This could also be used for registering a custom {\ttfamily Promise} implementation or subclass.

Your preference will be registered globally, allowing a single registration even if multiple versions of {\ttfamily any-\/promise} are installed in the N\+PM dependency tree or are using multiple bundled Java\+Script files in the browser. You can bypass this global registration in options\+:


\begin{DoxyCode}
require('../register')('es6-promise', \{Promise: require('es6-promise').Promise, global: false\})
\end{DoxyCode}


\subsubsection*{Library Usage}

To use any {\ttfamily Promise} constructor, simply require it\+:


\begin{DoxyCode}
var Promise = require('any-promise');

return Promise
  .all([xf, f, init, coll])
  .then(fn);


return new Promise(function(resolve, reject)\{
  try \{
    resolve(item);
  \} catch(e)\{
    reject(e);
  \}
\});
\end{DoxyCode}


Except noted below, libraries using {\ttfamily any-\/promise} should only use \href{https://developer.mozilla.org/en-US/docs/Web/JavaScript/Reference/Global_Objects/Promise}{\tt documented} functions as there is no guarantee which implementation will be chosen by the application author. Libraries should never call {\ttfamily register}, only the application user should call if desired.

\paragraph*{Advanced Library Usage}

If your library needs to branch code based on the registered implementation, you can retrieve it using `var impl = require(\textquotesingle{}any-\/promise/implementation'){\ttfamily , where}impl{\ttfamily will be the package name (}\char`\"{}bluebird\char`\"{}{\ttfamily ,}\char`\"{}when\char`\"{}{\ttfamily , etc.) if registered,}\char`\"{}global.\+Promise\char`\"{}{\ttfamily if using the global version on Node.\+js, or}\char`\"{}window.\+Promise\char`\"{}\`{} if using the browser version. You should always include a default case, as there is no guarantee what package may be registered.

\subsubsection*{Support for old Node.\+js versions}

Node.\+js versions prior to {\ttfamily v0.\+12} may have contained buggy versions of the global {\ttfamily Promise}. For this reason, the global {\ttfamily Promise} is not loaded automatically for these old versions. If using {\ttfamily any-\/promise} in Node.\+js versions versions {\ttfamily $<$= v0.\+12}, the user should register a desired implementation.

If an implementation is not registered, {\ttfamily any-\/promise} will attempt to discover an installed {\ttfamily Promise} implementation. If no implementation can be found, an error will be thrown on `require(\textquotesingle{}any-\/promise')\`{}. While the auto-\/discovery usually avoids errors, it is non-\/deterministic. It is recommended that the user always register a preferred implementation for older Node.\+js versions.

This auto-\/discovery is only available for Node.\+jS versions prior to {\ttfamily v0.\+12}. Any newer versions will always default to the global {\ttfamily Promise} implementation.

\subsubsection*{Related}


\begin{DoxyItemize}
\item \href{https://github.com/sindresorhus/any-observable}{\tt any-\/observable} -\/ {\ttfamily any-\/promise} for Observables. 
\end{DoxyItemize}