This package implements, in Java\+Script, the algorithms to convert a given Java\+Script value according to a given \href{http://heycam.github.io/webidl/}{\tt Web\+I\+DL} \href{http://heycam.github.io/webidl/#idl-types}{\tt type}.

The goal is that you should be able to write code like


\begin{DoxyCode}
const conversions = require("webidl-conversions");

function doStuff(x, y) \{
    x = conversions["boolean"](x);
    y = conversions["unsigned long"](y);
    // actual algorithm code here
\}
\end{DoxyCode}


and your function {\ttfamily do\+Stuff} will behave the same as a Web\+I\+DL operation declared as


\begin{DoxyCode}
void doStuff(boolean x, unsigned long y);
\end{DoxyCode}


\subsection*{A\+PI}

This package\textquotesingle{}s main module\textquotesingle{}s default export is an object with a variety of methods, each corresponding to a different Web\+I\+DL type. Each method, when invoked on a Java\+Script value, will give back the new Java\+Script value that results after passing through the Web\+I\+DL conversion rules. (See below for more details on what that means.) Alternately, the method could throw an error, if the Web\+I\+DL algorithm is specified to do so\+: for example {\ttfamily conversions\mbox{[}\char`\"{}float\char`\"{}\mbox{]}(NaN)} \href{http://heycam.github.io/webidl/#es-float}{\tt will throw a {\ttfamily Type\+Error}}.

\subsection*{Status}

All of the numeric types are implemented (float being implemented as double) and some others are as well -\/ check the source for all of them. This list will grow over time in service of the \href{https://github.com/dglazkov/html-as-custom-elements}{\tt H\+T\+ML as Custom Elements} project, but in the meantime, pull requests welcome!

I\textquotesingle{}m not sure yet what the strategy will be for modifiers, e.\+g. \href{http://heycam.github.io/webidl/#Clamp}{\tt {\ttfamily \mbox{[}Clamp\mbox{]}}}. Maybe something like {\ttfamily conversions\mbox{[}\char`\"{}unsigned long\char`\"{}\mbox{]}(x, \{ clamp\+: true \})}? We\textquotesingle{}ll see.

We might also want to extend the A\+PI to give better error messages, e.\+g. \char`\"{}\+Argument 1 of H\+T\+M\+L\+Media\+Element.\+fast\+Seek is not a finite floating-\/point value\char`\"{} instead of \char`\"{}\+Argument is not a finite floating-\/point value.\char`\"{} This would require passing in more information to the conversion functions than we currently do.

\subsection*{Background}

What\textquotesingle{}s actually going on here, conceptually, is pretty weird. Let\textquotesingle{}s try to explain.

Web\+I\+DL, as part of its madness-\/inducing design, has its own type system. When people write algorithms in web platform specs, they usually operate on Web\+I\+DL values, i.\+e. instances of Web\+I\+DL types. For example, if they were specifying the algorithm for our {\ttfamily do\+Stuff} operation above, they would treat {\ttfamily x} as a Web\+I\+DL value of \href{http://heycam.github.io/webidl/#idl-boolean}{\tt Web\+I\+DL type {\ttfamily boolean}}. Crucially, they would {\itshape not} treat {\ttfamily x} as a Java\+Script variable whose value is either the Java\+Script {\ttfamily true} or {\ttfamily false}. They\textquotesingle{}re instead working in a different type system altogether, with its own rules.

Separately from its type system, Web\+I\+DL defines a \href{http://heycam.github.io/webidl/#ecmascript-binding}{\tt \char`\"{}binding\char`\"{}} of the type system into Java\+Script. This contains rules like\+: when you pass a Java\+Script value to the Java\+Script method that manifests a given Web\+I\+DL operation, how does that get converted into a Web\+I\+DL value? For example, a Java\+Script {\ttfamily true} passed in the position of a Web\+I\+DL {\ttfamily boolean} argument becomes a Web\+I\+DL {\ttfamily true}. But, a Java\+Script {\ttfamily true} passed in the position of a \href{http://heycam.github.io/webidl/#idl-unsigned-long}{\tt Web\+I\+DL {\ttfamily unsigned long}} becomes a Web\+I\+DL {\ttfamily 1}. And so on.

Finally, we have the actual implementation code. This is usually C++, although these days \href{https://github.com/servo/servo}{\tt some smart people are using Rust}. The implementation, of course, has its own type system. So when they implement the Web\+I\+DL algorithms, they don\textquotesingle{}t actually use Web\+I\+DL values, since those aren\textquotesingle{}t \char`\"{}real\char`\"{} outside of specs. Instead, implementations apply the Web\+I\+DL binding rules in such a way as to convert incoming Java\+Script values into C++ values. For example, if code in the browser called {\ttfamily do\+Stuff(true, true)}, then the implementation code would eventually receive a C++ {\ttfamily bool} containing {\ttfamily true} and a C++ {\ttfamily uint32\+\_\+t} containing {\ttfamily 1}.

The upside of all this is that implementations can abstract all the conversion logic away, letting Web\+I\+DL handle it, and focus on implementing the relevant methods in C++ with values of the correct type already provided. That is payoff of Web\+I\+DL, in a nutshell.

And getting to that payoff is the goal of {\itshape this} project—but for Java\+Script implementations, instead of C++ ones. That is, this library is designed to make it easier for Java\+Script developers to write functions that behave like a given Web\+I\+DL operation. So conceptually, the conversion pipeline, which in its general form is Java\+Script values ↦ Web\+I\+DL values ↦ implementation-\/language values, in this case becomes Java\+Script values ↦ Web\+I\+DL values ↦ Java\+Script values. And that intermediate step is where all the logic is performed\+: a Java\+Script {\ttfamily true} becomes a Web\+I\+DL {\ttfamily 1} in an unsigned long context, which then becomes a Java\+Script {\ttfamily 1}.

\subsection*{Don\textquotesingle{}t Use This}

Seriously, why would you ever use this? You really shouldn\textquotesingle{}t. Web\+I\+DL is … not great, and you shouldn\textquotesingle{}t be emulating its semantics. If you\textquotesingle{}re looking for a generic argument-\/processing library, you should find one with better rules than those from Web\+I\+DL. In general, your Java\+Script should not be trying to become more like Web\+I\+DL; if anything, we should fix Web\+I\+DL to make it more like Java\+Script.

The {\itshape only} people who should use this are those trying to create faithful implementations (or polyfills) of web platform interfaces defined in Web\+I\+DL. 